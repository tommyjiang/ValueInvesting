\chapter{1957 年}

\section{1957 年的股票市场}

\begin{verseparallel}
  {
    \noindent In last year's letter to partners, I said the following: \\
  }
  {
    在去年给基金合伙人的信中,我写到:
  }
\end{verseparallel}

\begin{verseparallel}
  { \hspace{-0.9em}
    \textit{
    “My view of the general market level is that it is priced above intrinsic
    value. This view relates to blue-chip securities. This view, if accurate,
    carries with it the possibility of a substantial decline in all stock
    prices, both undervalued and otherwise. In any event I think the probability
    is very slight that current market levels will be thought of as cheap five
    years from now. Even a full-scale bear market, however, should not hurt the
    market value of our work-outs substantially.” \\
    }
  }
  { \hspace{-0.6em}
    \textit{
    我对目前市场的观点是股票的价格已经高于其内在价值。这个观点针对的是蓝筹股。如
    果这个观点是对的,有可能市场会有一个显著的下跌,不管是低估的股票还是其他股票。
    我觉得即使考虑未来五年的情况,目前的市场都不算便宜。但就算是一个完整的熊市,
    也不会对我们的 work-out 投资造成根本性的破坏。
    }
  }
\end{verseparallel}


\begin{verseparallel}
  {
    All of the above is not intended to imply that market analysis is foremost
    in my mind. Primary attention is given at all times to the detection of
    substantially undervalued securities. \\
  }
  {
    以上这些并不表明对市场的分析是我脑海中最重要的事,我们仍然把主要精力放在寻找
    低估的股票上。
  }
\end{verseparallel}

\begin{verseparallel}
  {
    The past year witnessed a moderate decline in stock prices. I stress the
    word `moderate' since casual reading of the press or conversing with those
    who have had only recent experience with stocks would tend to create an
    impression of a much greater decline. Actually, it appears to me that the
    decline in stock prices has been considerably less than the decline in
    corporate earning power under present business conditions. This means that
    the public is still very bullish on blue chip stocks and the general
    economic picture. I make no attempt to forecast either business or the stock
    market; the above is simply intended to dispel any notions that stocks have
    suffered any drastic decline or that the general market, is at a low level.
    I still consider the general market to be priced on the high side based on
    long term investment value. \\
  }
  {
    过去的一年股票价格出现了中等程度的下跌。我这里用的形容词是中等程度,因为偶尔
    翻阅报表的人和那些刚入股市的新手可能会觉得股票价格跌了很多。实际上,对我而言,
    股票价格下跌的程度,要小于企业盈利能力下降的程度。这说明目前大众对蓝筹股和经
    济形势仍然很乐观。我不想预测企业前景和股票市场,上面的分析只是想驳斥股价已经
    跌了很多且市场处于低位的观点。我仍然觉得目前的市场对于长期投资而言还处于高
    位。
  }
\end{verseparallel}

\section{1957 年的投资表现}

\begin{verseparallel}
  {
    The market decline has created greater opportunity among undervalued
    situations so that, generally, our portfolio is heavier in undervalued
    situations relative to work-outs than it was last year. Perhaps an
    explanation of the term `work-out' is in order. A work-out is an investment
    which is dependent on a specific corporate action for its profit rather than
    a general advance in the price of the stock as in the case of undervalued
    situations. Work-outs come about through: sales, mergers, liquidations,
    tenders, etc. In each case, the risk is that something will upset the
    applecart and cause the abandonment of the planned action, not that the
    economic picture will deteriorate and stocks decline generally. At the end
    of 1956, we had a ratio of about 70--30 between general issues and work-outs.
    Now it is about 85--15. \\
  }
  {
    市场的下跌创造了更多的低估机会,在市场低估时,我们会更多参与股票投资而
    非 work-out 投资。这里可能需要解释一下 work-out 的含义。Work-out 是指从公司的
    特定行为中获利,而非低估时买入股票从股价上升中获利。Work-out 投资包括公司的出
    售、并购、清算、要约收购等等。Work-out 投资的风险是公司的特定行为可能无法完
    成,而不是来自于经济形势的恶化和股价的下跌。1956 年底,我们股票投资
    和 work-out 投资的比例差不多是 70\% 和 30\%,现在 (1957 年底) 二者的比例差不多
    是 85\% 和 15\%。
  }
\end{verseparallel}

\begin{verseparallel}
  {
    During the past year we have taken positions in two situations which have
    reached a size where we may expect to take some part in corporate decisions.
    One of these positions accounts for between 10\% and 20\% of the portfolio
    of the various partnerships and the other accounts for about 5\%. Both of
    these will probably take in the neighborhood of three to five years of work
    but they presently appear to have potential for a high average annual rate
    of return with a minimum of risk. While not in the classification of
    work-outs, they have very little dependence on the general action of the
    stock market. Should the general market have a substantial rise, of course,
    I would expect this section of our portfolio to lag behind the action of the
    market. \\
  }
  {
    去年我们有两笔投资已经到达一定规模,可以影响所投资公司的决策。其中一笔占整个
    基金资金量的 10\% 到 20\%,另一笔占约 5\%。这两笔投资可能需要经过 3--5 年时间
    才能带来回报,但它们风险很低,回报率也不错。虽然这两笔投资不算 work-out,但它
    们和股票市场的相关性很小。因此如果股票市场出现大幅上涨,这两笔投资可能会相对
    大盘滞后,并不会马上上涨。
  }
\end{verseparallel}

\section{1957 年合伙人基金的表现}

\begin{verseparallel}
  {
    In 1957 the three partnerships which we formed in 1956 did substantially
    better than the general market. At the beginning of the year, the Dow-Jones
    Industrials stood at 499 and at the end of the year it was at 435 for a loss
    of 64 points. If one had owned the Averages, he would have received 22
    points in dividends reducing the overall loss to 42 points or 8.470\% for
    the year. This loss is roughly equivalent to what would have been achieved
    by investing in most investment funds and, to my knowledge, no investment
    fund invested in stocks showed a gain for the year. \\
  }
  {
    1957 年,我们在 1956 年成立的三个合伙人账户的表现都好于大盘。1957 年道琼斯开
    盘点数是 499,收盘点数是 435,跌了 64 点。考虑股利分红导致大盘跌了 22 点,大
    盘实际下跌 42 点,差不多 8.4\%。这个跌幅和大多数基金的跌幅差不多,而且据我所
    知,今年没有一只基金能取得正收益。
  }
\end{verseparallel}

\begin{verseparallel}
  {
    All three of the 1956 partnerships showed a gain during the year amounting
    to about 6.2\%, 7.8\% and 25\% on yearend 1956 net worth. Naturally a
    question is created as to the vastly superior performance of the last
    partnership, particularly in the mind of the partners of the first two. This
    performance emphasizes the importance of luck in the short run, particularly
    in regard to when funds are received. The third partnership was started the
    latest in 1956 when the market was at a lower level and when several
    securities were particularly attractive. Because of the availability of
    funds, large positions were taken in these issues. Whereas the two
    partnerships formed earlier were already substantially invested so that they
    could only take relatively small positions in these issues. \\
  }
  {
    1956 年,三只合伙人基金的净值分别增长了 6.2\%,7.8\% 和 25\%。最后一个基金表
    现比前两个好很多,前两个基金的投资人自然想知道表现不如第三个基金的原因。这恰
    恰说明短期业绩中运气的重要性,具体而言是基金成立时间的不同导致收益不同。第三
    只基金成立时间最晚,当时市场处于低位,有一些股票价格非常便宜。因此这只基金成
    立后,重仓了这些便宜的股票,而前两只基金之前已经建立了其他仓位,因此只能买入
    这些便宜股票很小的仓位。
  }
\end{verseparallel}

\begin{verseparallel}
  {
    Basically, all partnerships are invested in the same securities and in
    approximately the same percentages. However, particularly during the initial
    stages, money becomes available at varying times and varying levels of the
    market so there is more variation in results than is likely to be the case
    in later years. Over the years, I will be quite satisfied with a performance
    that is 10\% per year better than the Averages, so in respect to these three
    partnerships, 1957 was a successful and probably better than average, year. \\
  }
  {
    基本上所有的合伙人基金都会投资相同的股票,每只股票投入资金的比例也大致相同。
    然而在基金建仓阶段,由于基金收到资金的时间不同,市场价格水平不同,会导致短时
    间内不同资金业绩的差异,这种差异随时间变长会慢慢缩小。如果能跑赢大盘 10 个点,
    我就会感到非常满意。从三只基金的表现来看,1957 年收益不错,而且可能是收益高于
    平均水平的年份。
  }
\end{verseparallel}

\begin{verseparallel}
  {
    Two partnerships were started during the middle of 1957 and their results
    for the balance of the year were roughly the same as the performance of the
    Averages which were down about 12\% for the period since inception of the
    1957 partnerships. Their portfolios are now starting to approximate those of
    the 1956 partnerships and performance of the entire group should be much
    more comparable in the future. \\
  }
  {
    两只基金是 1957 年年中成立的,它们的表现和大盘差不多,从成立到现在跌了 12\%。
    现在这两只基金的持仓和 1956 年成立的基金类似,因此未来它们的表现会和 1956 年
    成立的基金类似,不会再像 1957 年一样有这么大的差异。
  }
\end{verseparallel}

\section{1957 年合伙人基金表现的说明}
\begin{verseparallel}
  {
    To some extent our better than average performance in 1957 was due to the
    fact that it was a generally poor year for most stocks. Our performance,
    relatively, is likely to be better in a bear market than in a bull market so
    that deductions made from the above results should be tempered by the fact
    that it was the type of year when we should have done relatively well. In a
    year when the general market had a substantial advance I would be well
    satisfied to match the advance of the Averages. \\
  }
  {
    某种程度上,我们在 1957 年能够跑赢大盘,是因为大多数股票表现不佳。我们基金在
    熊市中相对大盘的表现,会比牛市中更好。从上面的说法可以推断,1957 年我们的表现
    好于大盘,因为 1957 年大盘是熊市。如果大盘出现明显上涨,我希望我们基金的涨幅
    都够跟上大盘,这样我就会比较满意。
  }
\end{verseparallel}

\begin{verseparallel}
  {
    I can definitely say that our portfolio represents better value at the end
    of 1957 than it did at the end of 1956. This is due to both generally lower
    prices and the fact that we have had more time to acquire the more
    substantially undervalued securities which can only be acquired with
    patience. Earlier I mentioned our largest position which comprised 10\% to
    20\% of the assets of the various partnerships. In time I plan to have this
    represent 20\% of the assets of all partnerships but this cannot be hurried.
    Obviously during any acquisition period, our primary interest is to have the
    stock do nothing or decline rather than advance. Therefore, at any given
    time, a fair proportion of our portfolio may be in the sterile stage. This
    policy, while requiring patience, should maximize long term profits. \\
  }
  {
    我可以肯定地说,我们基金 1957 年底的资产配置比 1956 年底更好。这是因为市场更
    便宜,同时我们也有更多时间寻找那些需要耐心才能找到的低估的股票。上面我提到我
    们最大的一笔投资占了整个基金资金量的 10\% 到 20\%。我希望这笔投资能逐渐占到资
    金量的 20\%,但是加仓的行动急不得。很明显,在建仓阶段,我们希望股价保持不变或
    者下跌,而不希望它上涨。因此在这个阶段,我们投资组合中相当一部分投资的收益会
    比较差。这个投资策略需要一定的耐心,但能最大化我们的长期收益。
  }
\end{verseparallel}

\begin{verseparallel}
  {
    I have tried to cover points which I felt might be of interest and disclose
    as much of our philosophy as may be imparted without talking of individual
    issues. If there are any questions concerning any phase of the operation, I
    would welcome hearing from you.
  }
  {
    我在信中尽量将股东感兴趣的问题都谈到了,在避免谈论个股的同时也详细介绍了我们
    的投资理念。如果你对基金的运作有任何疑问,我都很乐意听听你的想法。
  }
\end{verseparallel}

%%% Local Variables:
%%% TeX-master: "../master"
%%% End:

\chapter{1958 年}

\section{1958 年的股票市场}

\begin{verseparallel}
  {
    \noindent A friend who runs a medium-sized investment trust recently wrote:
    `The mercurial temperament, characteristic of the American people, produced
    a major transformation in 1958 and `exuberant' would be the proper word
    for the stock market, at least''. \\
  }
  {
    一位掌管中等规模投资信托的朋友最近写到:“美国人善变的性格,导致了 1958 年市
    场的转变,繁荣这个词用来形容股市应该恰如其分。”
  }
\end{verseparallel}

\begin{verseparallel}
  {
    I think this summarizes the change in psychology dominating the stock market
    in 1958 at both the amateur and professional levels. During the past year
    almost any reason has been seized upon to justify “Investing” in the
    market. There are undoubtedly more mercurially-tempered people in the stock
    market now than for a good many years and the duration of their stay will be
    limited to how long they think profits can be made quickly and effortlessly.
    While it is impossible to determine how long they will continue to add
    numbers to their ranks and thereby stimulate rising prices, I believe it is
    valid to say that the longer their visit, the greater the reaction from it. \\
  }
  {
    我想上面的说法总结了业余投资者和专业投资者在 1958 年的心理变化,这种变化也
    主宰了 1958 年股票市场的表现。去年里几乎所有理由都被用于论证你应该投身股市进
    行投资。毫无疑问,现在市场上善变的人比之前一段时间更多,他们在股市中待多久,
    取决于他们觉得这样容易挣快钱的机会还能持续多久。尽管无法确定他们还会增加多少
    头寸从而进一步推高股价,但他们在股市中待的时间越长,股市受影响的程度也越高。
  }
\end{verseparallel}

\begin{verseparallel}
  {
    I make no attempt to forecast the general market --- my efforts are devoted
    to finding undervalued securities. However, I do believe that widespread
    public belief in the inevitability of profits from investment in stocks will
    lead to eventual trouble. Should this occur, prices, but not intrinsic
    values in my opinion, of even undervalued securities can be expected to be
    substantially affected. \\
  }
  {
    我不尝试预测股市 --- 我的精力都投入在寻找低估的股票上。但是我相信,目前广为流
    传的投资股市就能挣钱的说法最终会惹出麻烦。麻烦一旦出现,股票的价格而非价值会
    受到严重影响,即使那些低估的股票也同样难以幸免。
  }
\end{verseparallel}

\section{1958 年的投资表现}

\begin{verseparallel}
  {
    In my letter of last year, I wrote:
  }
  {
    在去年给股东的信中,我写到:\\
  }
\end{verseparallel}

\begin{verseparallel}
    { 
      \hspace{-0.9em}
      \textit{
      “Our performance, relatively, is likely to be better in a bear market
      than in a bull market so that deductions made from the above results
      should be tempered by the fact that it was the type of year when we should
      have done relatively will. In a year when the general market had a
      substantial advance, I would be well satisfied to match the advance of the
      averages.” \\
      }
    }
    {
      \hspace{-0.6em}
      \textit{
      我们基金在熊市中相对大盘的表现,会比牛市中更好。从上面的说法可以推断,1957
      年我们的表现好于大盘,因为 1957 年大盘是熊市。如果大盘出现明显上涨,我希望
      我们基金的涨幅都够跟上大盘,这样我就会比较满意。
      }
    }
\end{verseparallel}

\begin{verseparallel}
  {
    The latter sentence describes the type of year we had in 1958 and my
    forecast worked out. The Dow-Jones Industrial average advanced from 435 to
    583 which, after adding back dividends of about 20 points, gave an overall
    gain of 38.5\% from the Dow-Jones unit. The five partnerships that operated
    throughout the entire year obtained results averaging slightly better than
    this 38.5\%. Based on market values at the end of both years, their gains
    ranged from 36.7\% to 46.2\%. Considering the fact that a substantial
    portion of assets has been and still is invested in securities, which
    benefit very little from a fast-rising market, I believe these results are
    reasonably good. I will continue to forecast that our results will be above
    average in a declining or level market, but it will be all we can do to keep
    pace with a rising market.
  }
  {
    上一段的后面一句说的就是 1958 年的情况,这种情况也符合我之前的预测。道琼斯工
    业指数从 435 点涨到 583 点,再加上除权除息跌掉的近 20 点,去年道琼斯指数涨了
    38.5\%。五只合伙人基金的平均表现略高于这个值,收益率范围在 36.7\% 到 46.2\%
    之间。
  }
\end{verseparallel}
  
%%% Local Variables:
%%% TeX-master: "../master"
%%% End:

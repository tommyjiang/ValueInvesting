\chapter{伯克希尔的第一个五十年: 1965--2014}

\texttt{获得非凡的结果,并不一定要做非凡的事情。}

\begin{verseparallel}
  {
    \noindent Extraordinary is the only word that singularly captures the arc of
    Berkshire Hathaway’s fifty-year transformation under Warren Buffett’s
    control. The company that existed at the end of 2014 looked nothing like it
    did fifty years earlier despite bearing the same name. The struggling
    textile company that once formed the foundation of Berkshire Hathaway had
    cracked, leading to major structural issues that once rebuilt became a
    well-respected conglomerate. Its extraordinary transformation took place
    using what in hindsight were fairy ordinary and timeless basic principles of
    business. Applied step by step, year by year, and decade by decade, the
    ordinary was molded into the extraordinary. \\
  }
  {
    如果用一个词描绘在沃伦·巴菲特执掌下的伯克希尔·哈撒韦公司 50 年的转型发展脉
    络,那这个词非“非凡”莫属。这家存在于2014年底的公司看起来和50年前完全不同,尽管
    公司名称没有变。这家陷入困境的纺织公司曾经是伯克希尔·哈撒韦的基础,
    但它的经营出现了问题,导致出现了一些重大结构性问题。这家公司曾经重建并成为一
    家受人尊敬的大型企业集团。事后看来,该公司非凡的转型过程遵循的是极为普通而且
    永恒的基本商业原则。一步一步、一年一年、十年一年地应用,平凡被塑造成不平凡。
  }
\end{verseparallel}

%%% Local Variables:
%%% TeX-master: "../master"
%%% End:

\chapter{伯克希尔的第一个五十年: 1965--2014}

\begin{center}
  \texttt{'It is not necessary to do extraordinary things to get ekkjxtraordinary
    results.'} --- Warren Buffett

  \texttt{获得非凡的结果,并不一定要做非凡的事情。——沃伦·巴菲特}
\end{center}

\begin{verseparallel}
  {
    \noindent Extraordinary is the only word that singularly captures the arc of
    Berkshire Hathaway's fifty-year transformation under Warren Buffett's
    control. The company that existed at the end of 2014 looked nothing like it
    did fifty years earlier despite bearing the same name. The struggling
    textile company that once formed the foundation of Berkshire Hathaway had
    cracked, leading to major structural issues that once rebuilt became a
    well-respected conglomerate. Its extraordinary transformation took place
    using what in hindsight were fairy ordinary and timeless basic principles of
    business. Applied step by step, year by year, and decade by decade, the
    ordinary was molded into the extraordinary. \\
  }
  {
    如果用一个词描绘在沃伦·巴菲特执掌下的伯克希尔·哈撒韦公司 50 年的转型发展脉络,
    那这个词非“非凡”莫属。2014 年底的这家公司,尽管公司名称没有变,但和 50 年前
    的伯克希尔·哈撒韦公司相比已然脱胎换骨。曾经作为伯克希尔·哈撒韦公司基业的纺织
    公司倒下了,但重建后的伯克希尔,成为了广受尊敬的大型集团公司。从后视镜的视角
    来看,伯克希尔·哈撒韦公司的非凡转型过程遵循的其实是极为普通和永恒的基本商业原
    则。只是单纯一天天、一年年地应用这些原则,伯克希尔·哈撒韦公司从平凡被塑造成为
    不平凡。
  }
\end{verseparallel}

\begin{verseparallel}
  {
    Berkshire's transformation can easily be attributed to the one man who was
    the constant during this time. Yet Buffett is only part of the story---albeit a
    big part. The other man is Charlie Munger, whom Buffett credited as the
    architect of Berkshire Hathaway for his influence turning Berkshire's focus
    toward buying good businesses to hold for the long term. ``The blueprint he
    gave me was very simple. Forget what you know about buying fair
    businesses at wonderful prices; instead buy wonderful businesses at fair
    prices.'' \\
  }
  {
    伯克希尔的转型很容易归结于在此期间始终如一的巴菲特,但他其实只是这个故事的一
    部分——尽管占了很大篇幅。另一位是查理·芒格,巴菲特认为他是伯克希尔·哈撒韦的设
    计师,正是由于他的影响才让伯克希尔的注意力转向购买可以长期持有的优秀公
    司。“他给我的蓝图很简单。忘掉你之前了解的用便宜的价格购买一般的公司,去用合
    理的价格购买优秀的公司。”
  }
\end{verseparallel}

\begin{verseparallel}
  {
    These businesses were deceptively simple. Yes, they were in understandable
    industries such as insurance, retail, manufacturing, newspapers, and
    financing. But many shared a trait that most businesses only wish for: an
    economic moat or a sustainable competitive advantage. Berkshire's protective
    umbrella and autonomous operating philosophy were a system that maximized
    human potential and allowed businesses to flourish. \endnote{To the extent
    the businesses had the potential to flourish. No system could have stopped
    the few businesses that floundered, such as Dexter. (当然这需要公司本身具有
    繁荣发展的潜力,如果是德克斯特鞋业那样苦苦挣扎的公司,任何系统也没办法让它
    繁荣发展。)}. \\
  }
  {
    这些企业看起来实在太普通了。没错,它们都分布在普通人了解的行业,比如保险、零
    售、制造、报纸和金融。但这些企业拥有大多数企业都希望拥有的特质:经济护城河或
    可持续的竞争优势。伯克希尔对旗下企业的保护和让旗下企业自主经营的理念是一个系
    统,这个系统可以最大限度地发挥人的潜力,从而使旗下的企业繁荣发展。
  }
\end{verseparallel}

\begin{verseparallel}
  {
    Berkshire Hathaway's story also includes hundreds of owners and families
    that built up the many companies Berkshire came to own after shifting focus
    away from textiles. It includes the hundreds of thousands of employees that
    worked along with them. And it includes the hundreds of thousands of
    shareholders who made a long-term commitment to the company. Berkshire,
    then, is an amalgamation of these parts, all working together over a very
    long period. \\
  }
  {
    伯克希尔·哈撒韦公司的故事还包括数百个旗下公司的创始人以及他们的家族,这些人创
    立了众多的公司,伯克希尔从纺织业转型之后,对它们进行了收购。伯克希尔·哈撒韦公
    司的故事还包括数十万和他们一起工作的员工。同时这个故事也包括数十万长期对公司
    忠诚的股东。这样来看,伯克希尔是以上三个群体的综合,在很长一段时间里面他们都
    一起协同工作。
  }
\end{verseparallel}

\begin{verseparallel}
  {
    In 1965, when Buffett took over, Berkshire Hathaway did not make the Fortune
    500 list\endnote{Its last appearance was 1959 when it ranked 499th. (伯克希
    尔·哈撒韦最后一次出现在这个名单上是在 1959 年,排名第 499。)}. In 2014, it
    was number four behind Walmart, Exxon Mobil, and Chevron and ahead of
    Apple\endnote{Fortune magazine archives \: 2014 Fortune 500,
    \url{https://fortune.com/fortune500/2014/} \; 1965 Fortune 500,
    \url{https://archive.fortune.com/magazines/fortune/fortune500_archive/full/1965/401.html}
    (财富杂志存档:2014 财富 500 强名
    单\url{https://fortune.com/fortune500/2014/} 1965 财富 500 强名单
    \url{https://archive.fortune.com/magazines/fortune/fortune500_archive/full/1965/401.html})}.
    We can see the evolution of the company by examining its history in
    the broad decades-long periods outlined in this
    book. \\
  }
  {
    1965 年巴菲特入主伯克希尔·哈撒韦公司时,该公司并未跻身《财富》 500 强公
    司。2014年,它排在沃尔玛、埃克森美孚和雪佛龙之后,位列第四,紧随其后的是苹果
    公司。本书通过简要回顾伯克希尔·哈撒韦公司的历史,使读者看清它在过去几十年中的
    演变过程。
  }
\end{verseparallel}

\begin{section}{1965--1974}

\begin{table}[!htbp]
  \centering
\begin{center}
\begin{tabular}{cccc}
  \toprule
  \makecell[c]{(\$ millions) \\ (百万美元)} & 1974 & 1964 & \makecell[c]{Change \\ 同比增长} \\
  \midrule
  \makecell[c]{Revenues \\ 收入} & 101.5 & 50.0 & 103\% \\
  \makecell[c]{Pre-tax operating earnings \\ 税前营业利润} & 6.5 & 0.5 & 1,128\% \\
  \makecell[c]{Average float \\ 平均浮存金} & 79.1 & 0 & \makecell[c]{n/a \\ 不适用} \\
  \makecell[c]{Shareholders' equity \\ 所有者权益} & 88.2 & 22.1 & 298\% \\
  \makecell[c]{Book value per share \\ 每股净资产} & \$90.02 & \$19.46 & 363\% \\
  \bottomrule
\end{tabular}
\caption{Select data \\ 伯克希尔·哈撒韦公司的部分数据}
\end{center}
\end{table}

\begin{verseparallel}
  {
    After taking control of Berkshire in May 1965, Buffett quickly learned how
    difficult it was to operate a commodity business in a declining industry.
    Buffett's raw material was a dying textile company with \$22 million in net
    worth, no durable competitive advantage, and high capital costs. He quickly
    set to work redeploying as much of the available capital as possible into
    other businesses. \\
  }
  {
    在 1965 年 5 月获得伯克希尔的控制权后,巴菲特很快就意识到,在一个衰落的行业中
    经营实业是多么困难。巴菲特接手时的伯克希尔是一家垂死的纺织公司,净资产
    为 2200 万美元,没有持续的竞争优势,运营成本也很高。他迅速着手将尽可能多的可
    用资金重新安排到其他业务中。
  }
\end{verseparallel}

\begin{verseparallel}
  {
    Two seminal acquisitions occurred during this decade that shaped the future
    trajectory of Berkshire Hathaway. One was the acquisition of National
    Indemnity, which became the platform for future expansion into insurance.
    Buffett quickly grasped the value of low-cost liabilities in the form of
    float to fuel expansion in other areas. The beginnings of Berkshire's
    insurance activities provided valuable lessons on the importance of focusing
    on underwriting profitability above all else. The second influential
    acquisition was See's Candies. See's provided lessons on the value in
    buying great businesses for keeps. It set the bar very high for future
    acquisitions and was a marked contrast to its sister textile companies. \\
  }
  {
    在 1965--1974 这十年中,伯克希尔·哈撒韦进行了两次重大收购,决定了它未来的发
    展轨迹。其中之一是收购国家产险公司,这为伯克希尔未来进军保险领域提供了一个平
    台。巴菲特很快就抓住了保险浮存金这种低成本负债的核心价值,浮存金可以作为“燃
    料”支持公司在其他领域进行扩张。伯克希尔·哈撒韦公司涉足保险业务为我们提供的经
    验是,将承保的盈利能力放在所有事情的首位。另一个影响深远的是收购喜诗糖果公司。
    这次收购为我们提供的经验是,买入优秀的企业然后长期持有。与其他纺织公司不同,
    收购喜诗糖果之后,伯克希尔为后来的收购都设定了很高的标准。
  }
\end{verseparallel}

\begin{verseparallel}
  {
    Buffett made other important capital allocation decisions during this time.
    Berkshire purchased a newspaper, a bank, and made investments in marketable
    securities. One of those marketable securities was Blue Chip Stamps.
    Deploying the float in the shrinking trading stamps business, Buffett and
    Munger used Blue Chip Stamps as a platform to acquire See's, and eventually
    other good businesses before the core trading stamps business withered to
    almost nothing. \\
  }
  {
    在此期间,巴菲特还做出了其他一些重要的资本分配决策。伯克希尔买下了一家报纸和
    一家银行,并投资于股票。其中之一是购买了蓝筹印花公司的股票。巴菲特和芒格将浮
    存金投入在蓝筹印花公司不断萎缩的邮票交易业务之上,尽管核心的邮票交易业务最终
    化为乌有,但在这之前他们以蓝筹印花公司为平台收购了喜诗糖果,后来又收购了其他
    一些优秀的公司。
  }
\end{verseparallel}

\begin{verseparallel}
  {
    By the end of the decade, textiles had shrunk from the entirety of Berkshire
    Hathaway's business to about 30\% of consolidated revenues and just 5\% of
    total assets. The decline came from a combination of shrinking textile
    operations and expanding into new business lines. Textiles remained much
    longer than they probably would have had Buffett not chosen Berkshire as his
    investment vehicle. \\
  }
  {
    到 20 世纪 80 年代末,伯克希尔·哈撒韦公司的纺织业务明显收缩,收入占比从100\%
    下降到 30\% 左右,只占总资产的 5\%。纺织业务的占比下降来源于纺织业务自身的萎
    缩以及公司在其他行业的业务扩张。如果不是巴菲特收购了伯克希尔并将它作为他的投
    资工具,可能伯克希尔·哈撒韦的纺织业务维持的时间会长得多。
  }
\end{verseparallel}
\end{section}

\begin{section}{1975--1984}

\begin{table}[!htbp]
  \centering
  \begin{center}
    \begin{tabular}{cccc}
      \toprule
      \makecell[c]{(\$ millions) \\ (百万美元)} & 1984 & 1974 & \makecell[c]{Change \\ 同比增长} \\
      \midrule
      \makecell[c]{Revenues \\ 收入} & 729 & 101.5 & 618\% \\
      \makecell[c]{Pre-tax operating earnings \\ 税前营业利润} & 82.0 & 6.5 & 1,165\% \\
      \makecell[c]{Average float \\ 平均浮存金} & 253 & 79 & 220\% \\
      \makecell[c]{Shareholders' equity \\ 所有者权益} & 1,272 & 88.2 & 1,342\% \\
      \makecell[c]{Book value per share \\ 每股净资产} & \$1,109 & \$90.02 & 1,132\% \\
      \bottomrule
    \end{tabular}
    \caption{Select data \\ 伯克希尔·哈撒韦公司的部分数据}
  \end{center}
\end{table}

\begin{verseparallel}
  {
    The decade that ended in 1984 was marked by continued expansion of insurance
    operations and the acquisition of other non-insurance operating businesses.
    Written insurance premiums swelled 129\% from \$61 million in 1974 to \$140
    million in 1984 as Berkshire expanded operations organically and by forming
    numerous insurance companies. Its entry into reinsurance meant not only more
    float, but also longer-lived float. The big news of the decade was the
    acquisition of 36\% of GEICO.\@ Berkshire's share of GEICOs premium volume
    amounted to \$336 million—which dwarfed its homegrown operations. \\
  }
  {
    1975--1984 这十年,伯克希尔的保险业务持续扩张,同时还收购了一些非保险行业的公
    司。在这期间,伯克希尔的保险业有序扩张,成立了众多的保险公司,保费收入
    从 1974 年的 6100 万美元上涨 129\%,到 1984 年达到 1.4 亿美元。它进入了再保险
    行业,不仅获得了更多的浮存金,同时这些浮存金的使用期限也更长。这十年内最重
    要的事件是伯克希尔收购了 GEICO 36\% 的股份。伯克希尔持有的股份对应的 GEICO 保
    费高达 3.36 亿美元,与之相比,它自己的保险业务相形见绌。
  }
\end{verseparallel}

\begin{verseparallel}
  {
    This decade also witnessed the mergers of Diversified Retailing and Blue
    Chip Stamps into Berkshire. With Blue Chip Stamps came Wesco, yet another
    platform for expansion, this time into banking and insurance. Through Blue
    Chip Stamps, Berkshire acquired other non-insurance operations including The
    Buffalo News and Precision Steel. \\
  }
  {
    在这十年中,多元零售公司和蓝筹印花公司也并入了伯克希尔。蓝筹印花公司并入后,
    伯克希尔也就拥有了蓝筹印花公司旗下的威斯科公司,威斯科公司成为伯克希尔的另一
    个进军银行和保险领域的平台。以蓝筹印花公司为主体,伯克希尔收购了其他一些非保
    险行业的公司,包括布法罗新闻和精密钢铁公司。
  }
\end{verseparallel}

\begin{verseparallel}
  {
    The non-insurance companies acquired during this decade illustrated
    Buffett's appreciation of locally dominant businesses. While The Buffalo
    News experienced some initial threats, Buffett and Munger saw that one
    newspaper towns would create a protective moat allowing for superior returns
    on capital. Buffalo was a two-paper town at the time of the acquisition but
    became a one-paper town within five years with The Buffalo News the last one
    standing. Buffett also correctly identified Nebraska Furniture Mart as a
    dominant local business whose competitive advantage was created and
    reinforced by low margins coupled with huge volumes. \\
  }
  {
    巴菲特在这十年中收购了一些非保险行业的公司,这表明他很欣赏在当地具有统治力
    的企业。虽然《布法罗新闻报》在最初遇到了一些威胁,但巴菲特和芒格认为,如果一
    个小城市只有一家报纸的话,那么这就可以看做是这个报纸企业的一道护城河,可以实
    现更高的资本回报率。收购《布法罗新闻报》时,布法罗这个城市拥有两家报纸,但五
    年内它就只有《布法罗新闻报》一家报纸了。巴菲特还慧眼识珠地发现内布拉斯加州家
    具城也是一家在当地具有统治力的企业,它的竞争优势是通过薄利多销实现并且不断加
    强的。
  }
\end{verseparallel}

\begin{verseparallel}
  {
    Berkshire's investment activities during this period showed the value of
    taking partial ownership interests in wonderful companies. Gains from the
    investment portfolio were responsible for 58\% of the increase in
    Berkshire's net worth during this decade compared to just 11\% the prior
    decade. The investment portfolio reflected lessons learned elsewhere.
    Berkshire's success owning The Washington Post stock and The Buffalo News
    led it to invest in other media company stocks including American
    Broadcasting Companies, Inc., Capital Cities, and Time, Inc. Other stocks
    acquired during this time were mostly in simple, understandable businesses
    whose share prices had declined out of line with their underlying intrinsic
    values. \\
  }
  {
    伯克希尔在这十年间的投资活动展示了持有优秀公司部分股权的价值。在这十年间,投
    资组合获得的收益为伯克希尔贡献了 58\% 的净资产增长,而在上一个十年这个值仅
    为 11\%。伯克希尔持有的投资组合也反映了他们之前在其他地方学到的经验教训。因为
    持有《华盛顿邮报》 和《布法罗新闻报》等公司股票获得的成功,伯克希尔进一步投资
    了其他一些媒体公司,包括美国广播公司、资本城市公司和时代公司。在此期间,伯克
    希尔收购的大多都是简单易懂的公司,而且收购时其股价已经低于公司的内在价值。
  }
\end{verseparallel}

\begin{verseparallel}
  {
    Like a threadbare shirt, the last bit of Berkshire Hathaway's original
    business held on through this decade. By the end of 1984, though, the
    writing was on the wall for textiles. Almost immediately after acquiring
    Waumbec Mills, it was recognized as a mistake. The additional textile mill
    was eventually shuttered and faded along with the remainder of Berkshire's
    original textile operations. Textiles were no longer a profitable business.
  }
  {
    就像一件破旧的衬衫,伯克希尔·哈撒韦最初的纺织业务在这十年间仍然苟延残喘。但
    到 1984 年底,纺织行业已现不祥之兆。刚刚收购 Waumbec Mills 公司之后,巴菲特立
    即意识到这是个错误。后来,这家新纺织厂与伯克希尔·哈撒韦原有纺织业务的其余部分
    一起关闭。纺织行业不再是一个能挣钱的行业。
  }
\end{verseparallel}
\end{section}

\begin{section}{1985--1994}

\begin{table}[!htbp]
  \centering
  \begin{center}
    \begin{tabular}{cccc}
      \toprule
      \makecell[c]{(\$ millions) \\ (百万美元)} & 1994 & 1984 & \makecell[c]{Change \\ 同比增长} \\
      \midrule
      \makecell[c]{Revenues \\ 收入} & 3,847 & 729 & 428\% \\
      \makecell[c]{Pre-tax operating earnings \\ 税前营业利润} & 839 & 88 & 857\% \\
      \makecell[c]{Average float \\ 平均浮存金} 3,057 & 253 & 1,108\% \\
      \makecell[c]{Shareholders' equity \\ 所有者权益} & 11,875 & 1,272 & 834\% \\
      \makecell[c]{Book value per share \\ 每股净资产} & \$10,083 & \$1,109 & 809\% \\
      \bottomrule
    \end{tabular}
    \caption{Select data \\ 伯克希尔·哈撒韦公司的部分数据}
  \end{center}
\end{table}

\begin{verseparallel}
  {
    The decade that ended in 1994 marked when Berkshire hit its stride. During
    this decade, Berkshire perfected its understanding of insurance. It lost
    money in all but two of these years (1993--94) and used those lessons to
    engrain in the entire organization a philosophy of underwriting profitably
    first and foremost. Berkshire moved confidently into reinsurance and
    generated huge amounts of float to invest in marketable securities. Its
    strong balance sheet provided a double benefit. One was little restriction
    on where it could invest its float. Another was the ability to advertise its
    financial strength to attract additional reinsurance business. \\
  }
  {
    1985 年到 1994 年这十年是伯克希尔大步向前的十年。在这十年中,伯克希尔对保险行
    业的理解日臻完善。除了 1993 年和 1994 年这两年,伯克希尔在其他八年都出现了亏
    损,这也给了伯克希尔一个教训,使它深入了解了保险行业的理念,即承保时最重要的
    是确保盈利。伯克希尔·哈撒韦公司满怀信心地进入了再保险行业,利用浮存金大量投资
    股票。它强劲的资产负债表带来了双重好处。一是其浮存金的投资领域几乎不受限制,
    而是能够彰显公司的财务实力,从而吸引更多的再保险业务。
  }
\end{verseparallel}

\begin{verseparallel}
  {
    The major capital allocation decisions made during this decade were not
    complicated. Some of the businesses, such as Scott Fetzer and Fechheimer,
    were easy to understand but had been shunned by others. Berkshire provided a
    permanent home for these and many other simple businesses, and importantly
    allowed managers to operate with autonomy almost unheard of in corporate
    America. Buffett could do this because of a basic tenet he followed ``to go
    into business only with people whom I like, trust, and admire.'' \\
  }
  {
    伯克希尔在这十年间做出的重大资本配置决策并不复杂。其中一些公司,比如 Scott
    Fetzer 和 Fechheimer ,其实很容易理解,但其他人却没有看中这些公司。伯克希尔作
    为这些业务简单易懂的公司永远的后盾,而且最重要的是,它允许子公司的经理人享有
    在美国企业界几乎闻所未闻的自主权。巴菲特之所以能做到这一点,是因为他遵循一条
    基本原则:“只有我喜欢、信任和欣赏的人才能加入我们公司。”
  }
\end{verseparallel}

\begin{verseparallel}
  {
    The marketable securities portfolio was responsible for 68\% of the increase
    in Berkshire's net worth during this period. Here too the investments were
    not complicated and easy to fully understand in hindsight. Berkshire's
    experience with See's Candies led it to acquire a large stake in The Coca-
    Cola Company. Other investments during the decade included banks and
    consumer goods companies. Some of the investments, such as The Washington
    Post, ABC (which had merged with Capital Cities), and GEICO remained
    undisturbed and were viewed as near-permanent investments. Buffett's
    commentary to shareholders highlighted the large look-through earnings\endnote{Berkshire's share of retained earnings not paid out as dividends.
      (伯克希尔的未分配利润并没有进行分红。)} the portfolio represented to
    Berkshire and by extension its shareholders. \\
  }
  {
    在这十年间,伯克希尔的股票投资贡献了其净资产增长的 68\%。从事后诸葛亮的角度来
    看,伯克希尔的投资并不复杂,也很容易理解。伯克希尔收购喜诗糖果公司的经历促使
    它购买了可口可乐公司的大量股份。伯克希尔在这段时间的其他投资包括银行和消费品
    公司。它对《华盛顿邮报》、美国广播公司 (已经与 Capital Cities 合并)和美国政府雇员
    保险公司等企业的投资仍然维持原样,几乎等同于永久投资。巴菲特在给股东的信中重
    点强调了投资给伯克希尔和伯克希尔的股东贡献了巨大的透视盈余。
  }
\end{verseparallel}

\begin{verseparallel}
  {
    The decade was not without its mistakes. Buffett later pointed to the
    acquisition of Dexter Shoe as the worst in Berkshire's history because of
    the shares issued to acquire a business whose value quickly evaporated. Two
    of Berkshire's investments in convertible preferred stocks also caused
    trouble. USAir almost caused a loss. And its Salomon preferred caused a
    major distraction for Berkshire when Buffett temporarily took the helm of
    the investment bank to save it, something he had never done before. This
    short stint proved the advantages of allowing subsidiaries much autonomy. \\
  }
  {
    伯克希尔在这十年间也犯过错。正如巴菲特后来指出的,收购德克斯特鞋业是伯克希尔
    历史上最严重的一个错误,伯克希尔为了收购这家公司所付出的股票的价值迅速灰飞烟
    灭。与此同时,伯克希尔持有两家公司的可转换优先股,这两家公司也给伯克希尔惹了
    麻烦。USAir 几乎造成了损失,而伯克希尔更中意的所罗门公司让伯克希尔·哈撒韦不得
    不分出相当多的精力进行处理,巴菲特临时接管这家投资银行并试图挽狂澜于既倒,这
    是他之前从未做过的事。巴菲特的短暂任期也说明允许子公司拥有相当大自主权的好处。
  }
\end{verseparallel}

\begin{verseparallel}
  {
    Berkshire Hathaway ended 1994 free of the financial and managerial drag of
    the dying textile business, having shuttered the last of the operations in
    1986.
  }
  {
    1985--1994 这十年结束的时候,伯克希尔·哈撒韦终于摆脱了奄奄一息的纺织业在财务
    和管理上的拖累,它于 1986 年关闭了自己在纺织业的最后一个业务。
  }
\end{verseparallel}
\end{section}

\begin{section}{1995--2004}

\begin{table}[!htbp]
  \centering
  \begin{center}
    \begin{tabular}{cccc}
      \toprule
      \makecell[c]{(\$ millions) \\ (百万美元)} & 2004 & 1994 & \makecell[c]{Change \\ 同比增长} \\
      \midrule
      \makecell[c]{Revenues \\ 收入} & 74,382 & 3,847 & 1,834\% \\
      \makecell[c]{Pre-tax operating earnings \\ 税前营业利润} & 7,447 & 839 & 787\% \\
      \makecell[c]{Average float \\ 平均浮存金} 45,157 & 3,057 & 1,377\% \\
      \makecell[c]{Shareholders' equity \\ 所有者权益} & 85,900 & 11,875 & 623\% \\
      \makecell[c]{Book value per share \\ 每股净资产} & \$55,824 & \$10,083 & 454\% \\
      \bottomrule
    \end{tabular}
    \caption{Select data \\ 伯克希尔·哈撒韦公司的部分数据}
  \end{center}
\end{table}

\begin{verseparallel}
  {
    The 1995–2004 decade represented a rounding out and expansion of
    Berkshire's core operations. It also set the stage for the next phase of
    Berkshire's existence. During this decade Berkshire purchased the
    remaining half of GEICO it did not already own. It also acquired General
    Re by issuing shares. These insurance acquisitions were the final two pieces
    of Berkshire's insurance empire, which now included a major auto insurer,
    two reinsurance operations, and a host of smaller primary insurers that
    filled various niches of the insurance world. The acquisitions and organic
    growth swelled average float nearly fifteenfold to \$45 billion. \\
  }
  {
    1995--2004 这十年是伯克希尔·哈撒韦公司核心业务不断拓展和完善的十年,这也为伯
    克希尔的下一步发展奠定了基础。在这十年中,伯克希尔购买了 GEICO 的另一半股份。
    该公司还通过发行股票收购了通用再保险公司。这两个收购是伯克希尔·哈撒韦保险帝国
    的最后两块拼图,收购完成后,伯克希尔·哈撒韦保险帝国囊括了一家大型汽车保险公司、
    两个再保险公司,以及许多规模较小的一级保险公司,这些一级保险公司填补了保险行
    业的各种利基市场。通过收购和有序扩张,伯克希尔·哈撒韦公司的浮存金规模扩大
    了近 15 倍,高达 450 亿美元。
  }
\end{verseparallel}

\begin{verseparallel}
  {
    Berkshire acquired dozens of simple and essential non-insurance businesses
    during this decade. Many were shunned over emerging tech companies during
    the dot-com boom of the early 2000s. The numerous larger acquisitions in the
    non-insurance category were bolstered by many more bolt-on acquisitions.
    These fell under the direction of existing management and caused little to
    no additional work at headquarters. \\
  }
  {
    在这十年中,伯克希尔收购了几十家业务简单但地位重要的非保险行业公司。在 21 世
    纪初的互联网热潮中,伯克希尔并没有收购当时一些新兴科技公司。非保险行业公司的
    大规模收购,得益于更多的补强收购。这些收购都符合被收购公司管理层的意愿,因此
    对伯克希尔几乎甚至根本不需要做太多额外的工作。
  }
\end{verseparallel}

\begin{verseparallel}
  {
    The acquisition of MidAmerican set the stage for Berkshire's future. In the
    utility Berkshire obtained an outlet for its growing streams of cash. Future
    returns would be lower in more capital-intensive businesses like utilities,
    but the certainty attached to those capital outlays and ability to invest
    large sums of incremental capital made it an attractive platform.
    Berkshire's large base of taxable income elsewhere within the conglomerate
    provided an added advantage to its utility operations not available to its
    standalone peers. \\
  }
  {
    收购中美能源公司为伯克希尔的未来奠定了基础。公用事业公司为伯克希尔·哈撒韦源源
    不断的现金流提供了一个出口。在公用事业等资本密集型行业,未来的回报率会更低,
    但资本支出的确定性以及吸收大量资本的能力,使公用事业行业成为一个有吸引力的平
    台。与此同时,伯克希尔在其他行业获得的大量税前收入,为自己的公用事业公司提供
    了竞争对手无法获得的优势。
  }
\end{verseparallel}

\begin{verseparallel}
  {
    Berkshire ended the decade with \$40 billion in cash and not enough
    attractive outlets to invest in despite the frenzy of acquisition activity.
    The idle cash was a symptom of Berkshire's growing size and the shrinking
    universe of investment opportunities available to move the needle.
  }
  {
    这个十年将要过去的时候,伯克希尔手握 400 亿美元现金,尽管这十年的收购活动如火
    如荼,但吸引人的投资机会却越来越少。伯克希尔规模不断扩大,但投资机会越来越
    少,账面上留存的巨额现金很好地说明了伯克希尔面临的这种状况。
  }
\end{verseparallel}

\end{section}

\begin{section}{2005--2014}

\begin{table}[!htbp]
  \centering
  \begin{center}
    \begin{tabular}{cccc}
      \toprule
      \makecell[c]{(\$ millions) \\ (百万美元)} & 2014 & 2004 & \makecell[c]{Change \\ 同比增长} \\
      \midrule
      \makecell[c]{Revenues \\ 收入} & 194,673 & 74,382 & 162\% \\
      \makecell[c]{Pre-tax operating earnings \\ 税前营业利润} & 24,024 & 7,447 & 223\% \\
      \makecell[c]{Average float \\ 平均浮存金} 80,581 & 45,157 & 78\% \\
      \makecell[c]{Shareholders' equity \\ 所有者权益} & 240,170 & 85,900 & 180\% \\
      \makecell[c]{Book value per share \\ 每股净资产} & \$146,186 & \$55,824 & 162\% \\
      \bottomrule
    \end{tabular}
    \caption{Select data \\ 伯克希尔·哈撒韦公司的部分数据}
  \end{center}
\end{table}

\begin{verseparallel}
  {
    The decade that ended in 2014 may well have been the last where Berkshire
    was able to retain most of its earnings. Its rate of growth in book value
    per share slowed dramatically as cash accumulated without enough investment
    opportunities. Berkshire implemented a buyback policy and bought back its
    own shares on two occasions totaling \$1.7 billion. \\
  }
  {
    2005--2014 年这十年可能是伯克希尔·哈撒韦公司最后不分配自己利润的时光了。在没
    有足够投资机会的情况下,随着留存现金的不断积累,其每股账面价值的增长率显著放
    缓。伯克希尔选择回购自己公司的股票,总共回购了两次,总额为 17 亿美元。
  }
\end{verseparallel}

\begin{verseparallel}
  {
    The acquisitions made during this decade were numerous. It ended the decade
    with the Powerhouse Five, consisting of Iscar, Marmon, BNSF, Lubrizol, and
    Berkshire Hathaway Energy (formerly MidAmerican). Only the last was around
    in 2005. These businesses were easy to understand and provided a product or
    service sure to be needed long into the future. Two of them (BNSF and
    Lubrizol) were large public companies before joining Berkshire. With BNSF,
    Berkshire gained another utility-like operation that could take massive
    amounts of capital investment. These and numerous other acquisitions
    (including many bolt-on acquisitions) resulted in 70\% of
    the change in net worth coming from operations, up from 26\% the previous
    decade. \\
  }
  {
    这十年伯克希尔继续收购了众多公司,形成了“伯克希尔五巨头”,包
    括 Iscar、Marmon、伯灵顿北方圣太菲铁路运输公司、Lubrizol 和伯克希尔·哈撒韦能
    源公司(前中美能源公司),这些公司中只有伯克希尔·哈撒韦能源公司是在 2005 年收
    购的。这些公司的业务都很容易理解,而且它们提供的产品和服务都是伯克希尔在未来
    很长一段时间都需要的。其中两家(伯灵顿北方圣太菲铁路运输公司和 Lubrizol) 在
    并入伯克希尔之前已经是大型上市公司。有了伯灵顿北方圣太菲铁路运输公司,伯克希
    尔又多了一个类似公用事业的公司,正好这类公司需要大量资本投资。以上收购加上其
    他众多的收购(包括许多补强收购)导致伯克希尔 70\% 净资产变动来自 operations,而
    前十年这个数字仅为 26\%。
  }
\end{verseparallel}

\begin{verseparallel}
  {
    Berkshire's size and cash reserves had its advantages. During the Great
    Recession of the mid-2000s, it made very attractive fixed maturity
    investments at a time of very low interest rates when other businesses were
    short on cash. Berkshire also secured significant equity stakes along with
    them. Berkshire's unparalleled balance sheet strength created reinsurance
    opportunities other companies couldn't offer, including a single premium
    totaling \$7.1 billion. \\
  }
  {
    伯克希尔的规模和现金储备也给它带来了好处。在 2007--2008 年的大衰退期间,伯克
    希尔在非常低的利率环境下做了好几笔极具吸引力的固定期限投资,而其他企业此时却
    没有足够的现金。与此同时,伯克希尔公司还投资了大量股权。伯克希尔无与伦比的
    资产负债表实力使其能够获得其他公司无法获得的再保险机会,包括总额 71 亿美元的
    单一保单。
  }
\end{verseparallel}
\end{section}

\begin{section}{Concentrated Investments(集中投资)}

\begin{table}[!htbp]
  \centering
  \begin{center}
    \hspace*{-2cm}
    \begin{tabular}{cccccc}
      \toprule
      \makecell[c]{(Decade ended) \\ (每十年的结束时间)} & 1965--1974 & 1975--1984 & 1985--1994 & 1994--2004 & 2005--2014 \\
      \makecell[c]{Common stock portfolio: \\ 普通股投资组合} & & & & & \\
      \makecell[c]{Largest single common \\ stock investment (\% of portfolio): \\ 最大的普通股投资(仓位占比 \%)} & 23\% & 31\% & 34\% & 23\% & 23\% \\
      \makecell[c]{Top four common \\ stock investment (\% of portfolio): \\ 前四大普通股投资(仓位占比 \%)} & 47\% & 75\% & 57\% & 65\% & 59\% \\
      \makecell[c]{Top four \\ (\% of average equity at end of year): \\ 前四大普通股投资(资产占比 \%)} & 20\% & 79\% & 78\% & 30\% & 30\% \\
      \midrule
      \makecell[c]{Acquisitions (\% average equity \\ capital at time of purchase): \\ 收购时费用占当时总资产的比例} & & & & & \\
      \makecell[c]{Illinois National Bank \& Trust (1969) \\ (1969)} & 44\% & & & & \\
      \makecell[c]{National Indemnity (1967) \\ (1967)} & 28\% & & & & \\
      \makecell[c]{Buffalo News (1967) \\ (1977)} & & 15\% & & & \\
      \makecell[c]{Nebraska Furniture Mart (1983) \\ (1983)} & & 6\% & & & \\
      \makecell[c]{Scott Fetzer (1986) \\ (1986)} & & & 19\% & & \\
      \makecell[c]{Dexter Shoe (1993) \\ (1993)} & & & 4\% & & \\
      \makecell[c]{General Re (1998) \\ (1998)} & & & & 18\% & \\
      \makecell[c]{GEICO (1996) \\ (1996)} & & & & 12\% & \\
      \makecell[c]{BSNF (2010) \\ (2010)} & & & & & 18\% \\
      \makecell[c]{Heinz (2013) \\ (2013)} & & & & & 6\% \\
      \bottomrule
    \end{tabular}
    \caption{Select data \\ 伯克希尔·哈撒韦公司的部分数据}
  \end{center}
\end{table}



\begin{verseparallel}
  {
    Examining the broad arch, we can see that this half-century of Berkshire
    Hathaway's history consisted of a series of large, concentrated capital
    allocation decisions mixed with many smaller ones (see Table 8.6). At the
    end of the decade, Berkshire's equity investments were highly concentrated,
    with the Big Four (American Express, Coca-Cola, IBM, and Wells Fargo)
    accounting for 59\% of the portfolio. Berkshire's acquisitions were
    similarly concentrated. The largest acquisition in each decade represented
    no less than 15\% of equity capital at the time of acquisition. Berkshire's
    capital allocation strategy was one of patient opportunism. It made or held
    large partial interests in companies via the stock market and acquired
    successively larger companies as each decade went on. \\
  }
  {
    从宏观的角度,回顾伯克希尔·哈撒韦公司这半个世纪的历史,一系列大规模、集中投资
    与众多小规模投资相互交织(详见表8.6)。20 世纪 80 年代末,伯克希尔的股权投资
    高度集中,前四大投资(美国运通、可口可乐、IBM 和富国银行)的仓位高达 59\%。伯
    克希尔的收购活动也同样集中。在各个十年中的最大一次收购,在收购时都至少占了当
    时股东资本的 15\%。伯克希尔的投资和并购策略是耐心的等待机会。它通过股市买入或
    持有其他公司的大量股票,同时随着时间的推移,陆续收购越来越大的公司。
  }
\end{verseparallel}
\end{section}

\begin{section}{Rocket Fuel: Insurance Float(伯克希尔火箭腾空的燃料:保险浮存金)}
\begin{verseparallel}
  {
    Insurance float was perhaps the single most important factor driving
    Berkshire's growth over the first fifty years. Float produced both capital
    to deploy advantageously and substantial underwriting profits. Berkshire was
    in the insurance business forty-eight of the first fifty years of its modern
    existence. Discounting the partial first year in 1967, there was a negative
    cost of average float for twenty-seven years. In just eight of those years
    was Berkshire's cost of funds higher than that of the long-term US
    government bond. As time went on, Berkshire perfected its underwriting. This
    double benefit (increasing float and a loss experience that improved over
    time) resulted in substantial profits. Most of Berkshire's cumulative
    underwriting profits came in the last decade (see Table 8.7). \\
  }
  {
    在 1965--2014 这五十年间,保险浮存金或许是推动伯克希尔增长的最重要因素。保险
    浮存金既提可以作为长期优质资金用于投资,又可以产生可观的保费收入。伯克希尔
    在1965--2014 这五十年间的第三年就进入了保险行业。将1967年第一年的部分时间折现
    后,期间二十七年的浮存金平均成本为负值。在这五十年间,只有八年伯克希尔的资金
    成本高于美国长期政府债券的利率。随着时间的推移,伯克希尔完善了自己的承保业务。
    完善的承保带来了双重好处(增加了浮存金,同时随着时间的推移保险带来的亏损逐渐
    减少),从而带来了可观的利润。伯克希尔的大部分累计承保收入主要来自最后的十年,
    即 2005--2014 年(见表8.7)。
  }
\end{verseparallel}

\begin{table}[!htbp]
  \centering
  \begin{center}
    \begin{tabular}{cc}
      \toprule
      \makecell[c]{(\$ millions) \\ (百万美元)} & \\
      \midrule
      1968--1974 & (5) \\
      1975--1984 & (93) \\
      1985--1994 & (285) \\
      1995--2004 & (3,248) \\
      2005--2014 & 21,259 \\
      \bottomrule
    \end{tabular}
    \caption{Berkshire Hathaway pre-tax underwriting gain/ (loss) by decade \\ 每
      十年统计的伯克希尔·哈撒韦公司的税前承保收入/损失}
  \end{center}
\end{table}

{\color{red}{此处缺图 8.1}} \\

\end{section}

\begin{section}{A Self-Forged Anchor(一个自锁的锚)}

\begin{verseparallel}
  {
    Berkshire Hathaway was the victim of its own success. As the conglomerate
    grew larger and larger by retaining earnings, the universe of investment
    opportunities shrank dramatically. Compounding the problem was a market
    for businesses (either in part via the stock market or in whole) that
    became more efficient as the years went by. In Berkshire's early years,
    good companies were available for bargain prices. It bought the Illinois
    National Bank \& Trust Company and The Buffalo News at book value, and the
    discarded Scott Fetzer and Fechheimer at premiums that still produced
    going-in pre-tax returns in the mid-20\% range. During the decade ended in
    2014, it purchased a variety of businesses, some with underlying returns
    on capital well into the double digits. But the prices paid for these
    acquisitions cut the going-in returns down to the low double-digit or even
    single-digit range. \\
  }
  {
    伯克希尔·哈撒韦是自己成功发展的受害者。通过保留利润,这家大型企业规模变得越来
    越大,但投资机会却急剧减少。让问题变得更复杂的另一个因素,是随着时间的推移,
    企业的投资市场(不论是通过股市购买部分股票,或整体收购)变得越来越有效。在这
    五十年间的前一二十年,伯克希尔可以用低廉的价格买到优秀的企业。它以净资产价格
    收购了 The Illinois National Bank \& Trust Company 和 The Buffalo
    News,而Scott Fetzer 和 Fechheimer 的出价则低到买下它们后保费提供的税前回报率
    都高达 25\% 左右。而在 2005--2014 年这十年里,伯克希尔收购了各种各样的公司,
    其中一些公司的资本回报率甚至能达到了 20\% 以上,但伯克希尔的收购价格相对较高,
    从而使得长期资本回报率降低至 10\%-20\% 甚至只有几个百分点。
  }
\end{verseparallel}

\begin{verseparallel}
  {
    By charting a sample of Berkshire's acquisitions by purchase multiples and
    going-in returns, we can see both the value of different companies and a
    hint of the market situation when they were purchased (see Figure 8.2).
    Generally, the better the business was, the higher its price (as represented
    by purchase multiple paid compared to the company's underlying value). The
    return on capital of the underlying businesses (the company-level return)
    ranges widely. \\
  }
  {
    我们可以画出这样一个图,其横轴是伯克希尔所收购公司的长期资本回报率,纵轴是收
    购时付出的价格,通过这张图,我们可以看到不同公司的价值,以及在它们被收购当时
    的投资市场状况(见图8.2)。一般来说,公司越优质,其收购价格也就越高(这里的价
    格用收购价格与公司内在价值的比率来表示)。可以看到,伯克希尔旗下公司的资本回
    报率(公司级回报率)差异很大。
  }

\end{verseparallel}

{\color{red}{此处缺图 8.2}} \\

\begin{verseparallel}
  {
    The three major outliers are See's, Scott Fetzer, and Fechheimer. See's
    was one of Berkshire's earliest purchases and was made when markets were
    not as efficient. The low Scott Fetzer and Fechheimer purchase multiples
    reflected that Berkshire could act as a safe port amid the leverage buyout
    storm of the mid-1980s. By contrast, Lubrizol and Heinz were excellent
    companies earning great returns on capital, but the price Berkshire paid
    reflected the market's correct appraisal of that fact. \\
  }
  {
    三个主要的异常值分别是 See’s、Scott Fetzer 和 Fechheimer。See’s 是伯克希尔
    最早期的收购的之一,当时投资市场还相对无效,价格与价值偏差很大。而收购 Scott
    Fetzer和 Fechheimer 所付出的低价则表明,在 20 世纪 80 年代中期的杠杆收购风波
    中,伯克希尔可以充当一个避风港。相比之下,Lubrizol 和 Heinz 的资本回报率很出
    色,,但伯克希尔收购它们所支付的价格也反映了市场对这些优质公司的认可。
  }
\end{verseparallel}

\begin{verseparallel}
  {
    We can see the outperformance of Berkshire's book value and market values
    compared to the total return of the S\&P 500 decline over its fifty-year
    history (see Figure 8.3). Its advantage in compounding book value per share
    peaked at 20\% in the early 1980s and steadily declined to the single-digit
    percentage point range at the end of 2014. \\
  }
  {
    我们可以看到,伯克希尔的净资产和伯克希尔的市值,这二者相对于标准普尔 500 公司
    总回报率的超额表现,在这五十年中是逐步下降的(见图8.3)。伯克希尔的每股净资产
    超出标准普尔 500 公司的比例,在 20 世纪 80 年代初到达 20\% 的峰值,其后稳步下
    降,在 2014 年底的时候只有百分之几了。
  }
\end{verseparallel}

{\color{red}{此处缺图 8.3 和图 8.4}} \\

\begin{verseparallel}
  {
    Buffett was direct in his 2014 special letter discussing Berkshire's past,
    present, and future: ``The bad news is that Berkshire's long-term gains—
    measured by percentages, not by dollars—cannot be dramatic and will not come
    close to those achieved in the past 50 years. The numbers have become too
    big. I think Berkshire will outperform the average American company, but our
    advantage, if any, won't be great.'' \\
  }
  {
    2014年,巴菲特在一封给伯克希尔股东的特别信中,谈到了伯克希尔的过去、现在和未
    来,他的说法则更为直接:“一个坏消息:伯克希尔未来的长期收益——如果用百分比而
    不是美元来衡量——不会特别高,也不会接近我们在过去五十年中的水平。伯克希尔的规
    模太大了。我认为伯克希尔的表现将优于美国公司的平均水平,但即使我们优于平均水
    平,这个优势也不会太大。”
  }
\end{verseparallel}

\end{section}

\begin{section}{Broad Lessons(普适的经验)}

\begin{verseparallel}
  {
    Examining Berkshire's fifty-year history through 2014 several major
    lessons stand out: \\
  }
  {
    回顾伯克希尔从 1965 年到 2014 年这五十年的历史,我们可以学到一些重要的经验:
  }
\end{verseparallel}

{\color{green}{此处应为列表项 \\}}

\begin{verseparallel}
  {
    \textbf{Circle of competence}: Buffett and Munger, for the most part, stayed
    within their circle of competence. Building Berkshire used common sense and
    a focused strategy of choosing long-term businesses and investments with
    good economics that they understood well. \\
  }
  {
    \textbf{能力圈}:巴菲特和芒格基本上都待在他们自己的能力圈内。伯克希尔的成长
    壮大靠的是常识和这样一种策略,即专注选择他们理解很深、具有良好经济效益的业务,
    对它们进行长期投资。
  }
\end{verseparallel}

\begin{verseparallel}
  {
    \textbf{Business focus}: A central guiding principal was to focus on the
    underlying business. It did not matter whether the actual investment was a
    whole company, part of a company via stocks, or lending to a business via
    fixed income investments. A focus on the long-term economic characteristics
    of businesses was paramount to Berkshire's success. This included seeking
    businesses with strong economic moats. \\
  }
  {
    \textbf{关注公司本身}:一个中心指导原则是重点关注公司本身。无论是收购整个公司、
    还是买入公司的一部分股票,或是向公司放贷获取固定收益,这些都需要关注所投资公
    司本身。关注企业的长期经济特质对伯克希尔的成功至关重要。这包括寻找具有强大护
    城河的公司。
  }
\end{verseparallel}

\begin{verseparallel}
  {
    \textbf{Financial and operational conservatism}: Berkshire benefitted to an
    enormous degree from the float in insurance, and to a lesser extent the
    trading stamp business at Blue Chip Stamps (before it faded). Some debt was
    used when it was available on attractive terms, and later the utility
    businesses used debt as appropriate. Berkshire did not seek to increase
    returns by employing leverage. Its conservatism also extended to accounting.
    Countering the tendency toward optimism (and under-reserving) its loss
    reserving consistently overestimated insurance losses, which led to many
    years of favorable loss development. Berkshire took calculated risks with
    its capital and never suffered a large loss in relation to equity. It moved
    forward quicker by avoiding significant backward steps and common pitfalls. \\
  }
  {
    \textbf{财务和运营上的保守主义}:伯克希尔从保险行业的浮存金中受益很大,但从蓝
    筹印花公司(在其衰落前)的交易邮票业务中则受益较少。如果出现有吸引力的投资机
    会,伯克希尔有时会通过借债进行投资,伯克希尔后面收购的公共事业公司也会酌情举
    债。但伯克希尔并不会为了提高回报率而使用杠杆。这种财务上的稳健也延伸到了它的
    会计处理。与乐观主义(以及准备金储备不足)相反,伯克希尔的保险损失准备金始终
    高估承包可能带来的亏损,which led to 。伯克希尔精心计算它可能面临的风险,并为
    此准备了充足的资本,从未使公司的所有者权益遭受重大损失。伯克希尔之所以能更快
    地成长,在于它从未出现严重的倒退,同时避开了常见的陷阱。
  }
\end{verseparallel}

\begin{verseparallel}
  {
    \textbf{Opportunism}: Building Berkshire was an exercise in patience
    combined with opportunism and a reminder that opportunity cost matters.
    There was no grand strategy. Rather, Berkshire stood ready to make decisions
    as they arose. Each new investment was measured against what was already
    available. This included if and when to issue or repurchase its own shares.
    Related, Buffett maintained an entrepreneurial spirit that encouraged those
    within Berkshire to continually push the boundaries. Some of these
    entrepreneurial ventures failed, such as some of the early Home State
    operations, GEICO's credit card, and multiple attempts to expand See's
    beyond the West Coast. \\
  }
  {
    \textbf{耐心等待机会}:在伯克希尔的成长过程中,既要耐心等待机会,但有时机会成
    本也很重要。这其中没什么宏伟的策略。相反,伯克希尔只是时刻准备着,在机会出现
    时一把抓住它。每一个新的投资机会都会与目前所有可能的投资机会作对比,一种可能
    的投资是发行或回购伯克希尔自己的股份。与此相联系的是,巴菲特始终保持着一种创
    业精神,不断鼓励伯克希尔员工尝试突破边界,去做一些新业务。其中一些创业尝试失
    败了,比如 Home State 早期的运营,GEICO 发行的信用卡,以及多次尝试将 See's 的
    业务扩展到西海岸以外的其他地区。
  }
\end{verseparallel}

\begin{verseparallel}
  {
    \textbf{Concentrated investments}: As was discussed above, Berkshire
    combined its patience and its financial resources to make large investments
    when they became available. On more than one occasion, Buffett and Munger
    made investments that would not have been made if their objective was to
    avoid any risk. \\
  }
  {
    \textbf{集中投资}:正如之前所提到的,伯克希尔凭借足够的耐心和雄厚的财力,在机
    会出现时进行大规模投资。巴菲特和芒格抓住机会做了很多投资,如果他们的目标仅仅
    在于避免所有风险,那这些投资机会都会被他们错失。
  }
\end{verseparallel}

\begin{verseparallel}
  {

    \textbf{Conglomerate structure}: The conglomerate structure conferred many
    advantages to Berkshire. The two biggest were the ability to move capital
    seamlessly between operating segments and take advantage of tax benefits not
    available to others. Berkshire's purchase of several public companies meant
    those companies no longer had to spend time attracting capital for
    worthwhile projects or otherwise appease a sometimes shortsighted investor
    base. Buying for keeps also meant Berkshire could offer a permanent home for
    businesses built by individuals or families whose primary motivation was not
    to maximize the sale price. (Berkshire's conglomerate structure will be
    discussed in more detail in Part 10.) \\
  }
  {
    \textbf{伯克希尔公司的组织架构}:集团公司的组织架构给了伯克希尔许多优势,其中
    最大的两个优势是能够在不同的行业和领域之间无缝转移资金,并享受其他公司无法获
    得的税收优惠。伯克希尔收购了几家上市公司,这意味着这些公司再也不必花时间吸引
    资本,或者取悦时常目光短浅的股东群体。收购后相对长期并不转手还意味着,如果公
    司创始人或公司创始家族的主要动机不是最大化利润,而是经营好企业的话,伯克希尔
    可以为他们的企业提供一个永远的家。(伯克希尔的集团组织架构将在本书第 10 部分
    中详细讨论。)
  }
\end{verseparallel}

\begin{verseparallel}
  {
    \textbf{Autonomy}: Related to the conglomerate structure, the extreme
    autonomy given to operating managers meant the company could scale almost
    without limit and not become unwieldy. Autonomy had the added benefit of
    providing managers with a sense of ownership that motivated them outside of
    any monetary benefits. Berkshire maximized business potential by maximizing
    human potential.
  }
  {
    \textbf{自主经营权}:由于伯克希尔采用集团公司的组织架构,其子公司的经理人被赋
    予了极大的自主经营权,这意味着子公司几乎可以无限制地进行扩张,同时扩张并不会
    导致低效率。自主权的另一个好处是,它能让经理人觉得这是自己的公司,这会使他们
    充满干劲,任何金钱激励也无法比拟。伯克希尔通过最大化人的潜力来最大化自己的商
    业潜力。
  }
\end{verseparallel}

\begin{verseparallel}
  {
    The transformation Berkshire underwent during the 1965–2014 period was
    nothing short of breathtaking.
  }
  {
    伯克希尔在 1965年 到 2014 年期间的成功转型可谓令人惊叹。
  }
\end{verseparallel}

\end{section}

\theendnotes{}

\begin{chapter}{2015--2019}

\begin{table}[!htbp]
  \centering
  \begin{center}
    \hspace*{-2cm}
    \begin{tabular}{ccc}
      \toprule
       & \underline{2014} & \underline{2019} \\
      \makecell[c]{Business: \\ 行业:} & \makecell[c]{Insurance, utilities, railroad, numerous industrial, building, \\ and consumer products
      businesses, numerous service \\ and retailing businesses, major interest in a branded food \\
      product business, significant stakes in several \\ public companies \\ 保险业,公用事业,铁路,多种工业 \\} & 与左侧相同 \\
      \makecell[c]{Key managers: \\ 核心管理人员:} & \makecell[c]{Chairman \& CEO: Warren E.\@ Buffett; \\ Vice Chair: Charles T. Munger \\ 核心管理人员:主席 \& CEO:沃伦·巴菲特 \\ 副主席:查理·芒格} & 与左侧相同 \\
      \makecell[c]{Annual revenues: \\ 年度收入:} & \makecell[c]{\$195 billion \\ 1950 亿美元} & \makecell[c]{\$255 billion \\ 2550 亿美元} \\
      \makecell[c]{Stockholders' equity: \\ 所有者权益:} & \makecell[c]{\$240 billion \\ 2400 亿美元} & \makecell[c]{\$425 billion \\ 4250 亿美元} \\
      \makecell[c]{Book value per A share: \\ A 类股票每股净资产:} & \makecell[c]{\$146,186 \\ 146,186 美元} & \makecell[c]{\$261,417 \\ 261,417 美元} \\
      \makecell[c]{Float (average): \\ 平均浮存金} & \makecell[c]{\$81 billion \\ 810 亿美元} & \makecell[c]{\$126 billion \\ 1,260 亿美元} \\
      \bottomrule
    \end{tabular}
    \caption{Half decade snapshot: 2014--2019 \\ 五年快照:2014--2019}
  \end{center}
\end{table}

\begin{table}[!htbp]
  \centering
  \begin{center}
    \hspace*{-2cm}
    \begin{tabular}{ccc}
      \toprule
      \multicolumn{3}{l}{\makecell[l]{\textit{Major capital allocation decisions: } \\ \textit{主要资本决策:}}} \\
      \multicolumn{3}{l}{\makecell[l]{1. Acquired Van Tuyl Automotive Group for \$4.2 billion cash (2015). \\ 1. 以 42 亿美元现金收购 Van Tuyl Automotive Group(2015 年)。}} \\
      \multicolumn{3}{l}{\makecell[l]{2. Invested an additional \$5 billion in Heinz to acquire Kraft Foods Group (2015). \\ 2. 额外投资 Heinz 50 亿美元用于收购 Kraft Foods Group(2015 年)。}} \\
      \multicolumn{3}{l}{\makecell[l]{3. Acquired Precision Castparts Corp. (PCC) for \$32.6 billion cash (2016). \\ 3. 以 326 亿美元收购 Precision Castparts Corp(2016 年)。}} \\
      \multicolumn{3}{l}{\makecell[l]{4. Acquired Duracell from Proctor \& Gamble in a cash-rich split-off transaction for \$4.2 billion, \\ gross of \$1.8 billion acquired cash (2016). \\ 4. 以 42 亿美元从 Proctor \& Gamble 手中收购 Duracell,采用现金充裕剥离交易,\\ 收购付出的现金总额为 18 亿美元 (2016 年)。}} \\
      \multicolumn{3}{l}{\makecell[l]{5. Invested \$8.9 billion in the four largest US airlines. \\ 5. 向美国四大航空公司投资 89 亿美元。}} \\
      \multicolumn{3}{l}{\makecell[l]{6. Acquired 38.6\% initial stake in Pilot Flying J for \$2.8 billion (2017). \\ 6. 以 28 亿美元收购 Pilot Flying J 38.6\% 的原始股(2017 年)。}} \\
      \multicolumn{3}{l}{\makecell[l]{7. Sold entire investment in IBM for approximately \$13 billion (2017). \\ 7. 以大约 130 亿美元出售之前买入的所有 IBM 股票(2017 年)。}} \\
      \multicolumn{3}{l}{\makecell[l]{8. Acquired Medical Liability Mutual Insurance Company for \$2.5 billion cash (2018). \\ 8. 以 25 亿美元收购 Medical Liability Mutual Insurance Company(2018 年)。}} \\
      \multicolumn{3}{l}{\makecell[l]{9. Acquired a significant stake in Apple, Inc.: \$6.7 billion (2016), \$14 billion (2017) and \$15 billion (2018) \\ for a total investment of \$36 billion or 5.4\% of the company. \\ 9. 购买了大量的苹果公司股票: 2016 年买入 67 亿美元,2017 年买入 140 亿美元,2018 年买入 \\ 150 亿美元,总共买入 360 亿美元,持有苹果公司 5.4\% 的股票。}} \\
      \multicolumn{3}{l}{\makecell[l]{10. Sold 81\% interest in Applied Underwriters for \$920 million (2019). \\ 10. 以 9.2 亿美元出售所持有 Applied Underwriters 股份的 81\%(2019 年)。}} \\
      \multicolumn{3}{l}{\makecell[l]{11. Invested \$10 billion in Occidental Petroleum preferred stock. \\ 11. 买入 100 亿美元 Occidental Petroleum 公司的优先股。}} \\
      \multicolumn{3}{l}{\makecell[l]{12. Invested over \$71 billion in capital expenditures (\$32 billion more than depreciation expense) (2015--2019). \\ 12. 花费超过 710 亿美元用于资本开销(折旧超过 320 亿美元)(2015--2019 年)。}} \\
      \multicolumn{3}{l}{\makecell[l]{13. Repurchased \$6.4 billion of its own stock (2018--2019). \\ 13. 回购 64 亿美元股票(2018--2019 年)。}} \\
      \bottomrule
    \end{tabular}
    \caption{Half decade snapshot: 2014--2019 \\ 五年快照:2014--2019 续1}
  \end{center}
\end{table}

\begin{table}[!htbp]
  \centering
  \begin{center}
    \hspace*{-2cm}
    \begin{tabular}{ccc}
      \toprule
      \multicolumn{3}{l}{\makecell[l]{\textit{Noteworthy events: } \\ \textit{大事记:}}} \\
      \multicolumn{3}{l}{\makecell[l]{1. The Insurance Group breaks its 14-year record of consecutive underwriting profits with a pre-tax \\ loss of \$3.2 billion. GEICO reports its first underwriting loss since 2000. (2017) \\ 1. The Insurance Group 连续十四年承保盈利终结,税前亏损 32 亿美元,GEICO 自 2000 年以来 \\ 首次出现亏损(2017 年)。}} \\
      \multicolumn{3}{l}{\makecell[l]{2. Completes the largest retroactive reinsurance deal in history with a \$10.2 billion premium with AIG (2017). \\ 2. 与 AIG 签订了史上最大的有追溯效力的再保险合同,保费高达 102 亿美元(2017 年)。}} \\
      \multicolumn{3}{l}{\makecell[l]{3. Buffett wins a ten-year bet that an index fund will beat a group of hedge funds (2017). \\ 3. 巴菲特打赌单个指数基金就能战胜一组对冲基金的组合,他赌赢了。(2017 年)。}} \\
      \multicolumn{3}{l}{\makecell[l]{4. Congress passed the Tax Cuts and Jobs Act of 2017 which reduces the US Corporate tax rate from 35\% to 21\%. \\ 4. 美国国会 2017 年通过减税与就业法案,美国企业的税率从 35\% 降至 21\%。}} \\
      \bottomrule
    \end{tabular}
    \caption{Half decade snapshot: 2014--2019 \\ 五年快照:2014--2019 续2}
  \end{center}
\end{table}

\begin{verseparallel}
  {

    This following tables have been omitted from the ebook version because
    formatting issues would have rendered them unreadable. The reader is welcome
    to download a pdf version of the omitted tables and bonus material at
    \url{brkbook.com}. \\
  }
  {
    下面两个表格如果在电子书中排版,其可读性很差,因此电子书中略去了这两个表格。
    欢迎读者到 \url{brkbook.com} 网站下载这两个表格的 pdf 版本和其他补充材料。
  }
\end{verseparallel}

\begin{section}{Introduction(本章导言)}
\begin{verseparallel}
  {
    Berkshire Hathaway entered its sixth decade of transformation in uncharted
    territory. It wasn't just that one man held the reins for so long, though
    that is noteworthy. A conglomerate of Berkshire's size simply hadn't
    existed before. How would the playbook change as capital began to accumulate
    faster than it could profitably be allocated? Since this book is being
    finalized in 2020, we must wait to see the longer story unfold. The first
    five years of the 2015–2024 decade looked much like the decade that preceded
    it. In short, Berkshire continued to make the most intelligent decisions at
    any time. \\
  }
  {
    伯克希尔迎来了它在前无来者领域探索的第六个十年。所谓的前无来者不仅仅指一个人
    掌舵一个企业如此之久,尽管这一点也值得大书特书,更是指以前从没出现过规模达到
    伯克希尔这个量级的公司。当钱太多而投资机会又太少时,伯克希尔将会如何应对?由
    于本书在 2020 年定稿,我们也只有等待未来的画卷徐徐展开后才能知晓这个问题的答
    案。2015--2020 这五年和 2005--2014 这十年非常像。简而言之,伯克希尔仍然做出了
    最明智的决定。
  }
\end{verseparallel}

\begin{verseparallel}
  {
    The conglomerate generated huge amounts of capital between 2015 and
    2019 thanks to retained earnings and the effects of compounding. Berkshire
    continued to benefit from a playbook that included options to allocate
    capital into wholly-owned businesses or stocks. It found some smart uses of
    capital in the face of sky-high business valuations and an ever-advancing
    stock market fueled by continued low interest rates. One major acquisition
    materialized, as did a host of smaller bolt-on acquisitions that soaked up
    some capital. So too did Berkshire's partnership with 3G Capital, which
    allowed it to add another household name to the roster of businesses it
    owned or controlled.\\
  }
  {
    伯克希尔在 2015 到 2019年间积累了巨额资本,这要归功于留存收益和复利效应。伯克
    希尔·哈撒韦继续执行向全资子公司投入资金或购买股票的投资策略,这也使它持续增长。
    面对估值高高在上的上市公司以及持续低利率环境下不断上涨的股市,伯克希尔的资本
    仍然有相对明智的用途。它完成了一次大规模的收购,同时还用一些资本完成了一系列
    补强收购。此外,伯克希尔与 3G Capital 进行了合作,使它所拥有或控股的一系列子
    公司名单上,又多了一个家喻户晓的名字。
  }
\end{verseparallel}

\begin{verseparallel}
  {
    Berkshire's future in a post-Buffett world became clearer during this time.
    It restructured management and promoted two long-time lieutenants to vice
    chairmen. However, the question of who would succeed Buffett as CEO
    remained. The gusher of cash found a partial relief valve in an expanded
    share repurchase program. Berkshire modified the criteria and returned a
    modicum of capital to shareholders in the form of buybacks. Would Berkshire
    have opportunity to return more cash through buybacks? Would it institute a
    dividend? These questions remained as cash built to a record \$128 billion at
    year-end 2019 despite best efforts to use it. \\
  }
  {
    在这几年间,后巴菲特时代的伯克希尔的前景变得更加明朗。伯克希尔重组了管理层,
    将两名长期任职的助手提升为副主席。然而,谁将接替巴菲特担任首席执行官依然悬而
    未决。更多的股份回购为源源不断膨胀的现金提供了一个宣泄的出口。伯克希尔没有继
    续坚持不分红,通过回购向股东派发了一部分现金。伯克希尔还有机会通过回购来派发
    更多现金吗?它会直接分红吗?这些问题都是伯克希尔未来需要考虑的,尽管想尽各种
    办法处理现金,但它的现金规模在 2019 年底仍然达到了创纪录的 1280 亿美元。
  }
\end{verseparallel}

\begin{verseparallel}
  {
    One fact remained very clear. Warren Buffett and Charlie Munger weren't
    done shaping Berkshire Hathaway.
  }
  {
    只有一件事是确凿无疑的,巴菲特和芒格还在持续不断地改造着伯克希尔。
  }
\end{verseparallel}

\begin{table}[!htbp]
  \centering
  \begin{center}
    \begin{tabular}{cccccc}
      \toprule
      & \underline{2015} & \underline{2016} & \underline{2017} & \underline{2018} & \underline{2019}  \\
      \makecell[c]{BRK book value per share -\% change \\ 伯克希尔每股净资产同比增长} & 6.4\% & 10.7\% & 23.0\% & 0.4\% & 23.0\% \\
      \makecell[c]{BRK market value per share -\% change \\ 伯克希尔每股价格同比增长} & (12.5\%) & 23.4\% & 21.9\% & 2.8\% & 11.0\% \\
      \makecell[c]{S\&P 500 total return \\ 标普 500 回报率} & 1.4\% & 12.0\% & 21.8\% & (4.4\%) & 31.5\% \\
      \makecell[c]{US GDP Growth (real \%) \\ 美国 GDP 实际增长率} & 3.1\% & 1.7\% & 2.3\% & 3.0\% & 2.2\% \\
      \makecell[c]{10-year Treasury Note (year-end \%) \\ 十年期美国国债收益率} & 2.2\% & 2.5\% & 2.4\% & 2.8\% & 1.9\% \\
      \makecell[c]{US inflation (avg.\ annual \%) \\ 美国通货膨胀率(年平均值)} & 0.1\% & 1.3\% & 2.1\% & 2.4\% & 1.8\% \\
      \makecell[c]{US unemployment (avg.\ annual \%) \\ 美国失业率(年平均值)} & 5.3\% & 4.9\% & 4.3\% & 3.9\% & 3.7\% \\
      \bottomrule
    \end{tabular}
    \caption{Select information 2015--2019 \\ 伯克希尔 2015--2019 的部分数据}
  \end{center}
\end{table}

\end{section}

\begin{section}{2015}
\begin{verseparallel}
  {
    Berkshire's gain in net worth during 2015 amounted to \$15.4 billion, an
    increase of 6.4\%. This was significantly ahead of the 1.4\% gain recorded
    for the S\&P 500. But in 2014, Buffett had switched his preferred yardstick
    to Berkshire's change in market value per share, which fell 12.5\%. These
    were two competing data points. Which to believe? Buffett was confident the
    market value metric would prevail over time. He knew Berkshire's share
    price, like that of the market in general, would rise and fall but
    eventually settle close to intrinsic value. A single year was not enough
    data to draw any definitive conclusion. Buffett preferred to look at
    Berkshire's progress building normalized earnings power (earnings excluding
    any gains or losses from marketable securities or derivatives) to evaluate a
    single year, and he thought Berkshire had a good year on that front. \\
  }
  {
    2015年,伯克希尔的净资产增长了 154 亿美元,同比增长 6.4\%,比标准普尔 500 高
    出 1.4\%。但 2014 年巴菲特把自己的衡量标准改为伯克希尔每股价格的同比涨幅,而
    当年伯克希尔的股价跌了 12.5\%。净资产和市值都是很有代表性的数据,那么哪一个更
    可信?巴菲特认为,随着时间的推移,市值指标将会胜出。他觉得伯克希尔的股价和所
    有公司的股价一样会有起伏波动,但最终会稳定在接近其内在价值的水平上。当然,某
    一年的数据并不足以得出任何有说服力的结论。巴菲特倾向于用伯克希尔的正则盈利能
    力(不包括股票或衍生品带来的收益或亏损)来评估这一年的业绩,他认为伯克希
    尔 2015 年在这方面表现还不错。
  }
\end{verseparallel}

\begin{verseparallel}
  {
    The Powerhouse Five, the largest non-insurance businesses (comprised of
    Berkshire Hathaway Energy, BNSF, Iscar, Lubrizol, and Marmon), reported
    record earnings. That included a turnaround by BNSF, which earned Buffett's
    praise in 2015 after disappointing the prior year. The dozens of other
    non-insurance businesses increased their earnings too. Insurance turned in
    its thirteenth consecutive year of underwriting profits and again increased
    float. Large capital expenditures and many bolt-on acquisitions increased
    the earnings power of Berkshire's existing businesses. The partnership with
    3G Capital expanded again when Heinz merged with Kraft to create a consumer
    brand giant. Additional capital was put to work in equity securities. \\
  }
  {
    伯克希尔在非保险行业的五巨头(包括伯克希尔·哈撒韦能源公
    司、BNSF、Iscar、Lubrizol 和 Marmon)录得创纪录的盈利。这包括 BNSF 经营状况的
    大幅改善,巴菲特在 2015 年称赞了这家公司,而就在 2014 年巴菲特还对它非常失望。
    其他几十家非保险公司的盈利也有所增长。保险业则连续第十三年收获承保业务的盈利,从
    而使浮存金继续增长。巨额的资本投入和大量补强收购提高了伯克希尔现有业务的盈利
    能力。伯克希尔与 3G Capital 继续合作,推动 Heinz 与 Kraft 合并,形成了一个消
    费品牌巨头,而伯克希尔其余的资本则继续投资股票。
  }
\end{verseparallel}

\begin{verseparallel}
  {
    In Buffett's Chairman's letter he provided an update on the two quantitative
    factors he thought useful for estimating Berkshire's intrinsic value. For
    the first time he included underwriting profit in the per-share operating
    earnings figure. \endnote{Underwriting earnings per share amounted to
      \$1,118 in 2015 and increased the intrinsic value estimate by 4\%. Doing
      so also reduced the change in estimated intrinsic value if the \$1,624
      underwriting earnings per share in 2014 are included that year.(2015 年伯
      克希尔每股的承保盈利是 1,118 美元,其内在价值同比增长 4\%。如果将 2014 年每
      股承保盈利 1,624 美元考虑在内,其内在价值的同比增长值会小一些。)} His
    reasoning was the underwriting business had changed substantially; earnings
    were more stable than a decade or two ago and were less heavily influenced
    by catastrophe coverage. Still, Buffett was quick to point out that an
    underwriting loss remained possible as the super cat
    business hadn't gone away; it just diminished in relation to other business. \\
  }
  {
    在巴菲特写给伯克希尔股东的信中,他更新了自己对以上两个对衡量伯克希尔内在价值
    有价值的指标的看法。他头一次将承保盈利加入到每股营业利润中。他给出的理由是,
    承保行业发生了翻天覆地的变化,与前一二十年相比,盈利变得更稳定,同时也更不容
    易受到极端黑天鹅事件的影响。但他又立即话锋一转,指出导致承保亏损的极端黑天鹅
    事件仍然有可能发生,只是由于其他业务的增长,单个黑天鹅事件的影响越来越小了。
  }
\end{verseparallel}

\begin{table}[!htbp]
  \centering
  \begin{center}
    \hspace*{-3cm}
    \begin{tabular}{ccccc}
      \toprule
      & \multicolumn{2}{c}{ \makecell[c]{With insurance underwriting \\ 计入承保收入}} & \multicolumn{2}{c}{ \makecell[c]{W/out insurance underwriting \\ 不计入承保收入}} \\
      \makecell[c]{\textit{Per share (A-equivalent):} \\ \textit{A 类股每股平均值}} & \underline{2015} & \underline{2014} & \underline{2015} & \underline{2014} \\
      \makecell[c]{Investments (Kraft Heinz at market) \\ 投资(Kraft Heinz 市值)} & \$159,794 & \$140,123 & \$159,794 & \$140,123 \\
      \makecell[c]{Pre-tax operating earnings (ex.\ investment income) \\ 税前营业利润(去掉投资收益)} & \$12,304 & \$12,471 & \$11,186 & \$10,847 \\
      \makecell[c]{Estimated value (investments + 10x operating earnings) \\ 估算价值(投资收益 + 10 倍营业利润)} & \$282,834 & \$264,832 & \$271,654 & \$248,593 \\
      \makecell[c]{Year-end share price \\ 年底股价} & \$197,800 & \$226,000 & \$197,800 & \$226,000 \\
      \makecell[c]{Year-end book value per share \\ 年底每股净资产} & \$155,501 & \$146,186 & \$155,501 & \$146,186 \\
      \makecell[c]{Price/book \\ 股价与每股净资产比值} & 1.27x & 1.55x & 1.27x & 1.55x \\
      \makecell[c]{Value/book \\ 估算价值与每股净资产比值} & 1.82x & 1.81x & 1.75x & 1.70x \\
      \makecell[c]{Change in estimated value \\ 估算价值同比增长} & 6.8\% & & 9.3\% & \\
      \makecell[c]{Change in share price \\ 股价同比增长} & (12.5\%) & & (12.5\%) & \\
      \bottomrule
    \end{tabular}
    \caption{Select information 2015--2019 \\ 伯克希尔 2015--2019 的部分数据}
  \end{center}
\end{table}

\begin{subsection}{Insurance(保险业)}
\begin{verseparallel}
  {
    The Insurance Group delivered a \$1.8 billion pre-tax underwriting gain in
    2015 in addition to increasing year-end float by 4.5\% to \$87.7 billion.
    Each major insurance segment was profitable, though not without their unique
    challenges. \\
  }
  {
    2015年,伯克希尔的保险公司们实现了 18 亿美元的税前承保盈利,从而使伯克希尔的
    在 2015 年底时的浮存金同比增长 4.5\%,达到 877 亿美元。尽管伯克希尔旗下的每家
    大型保险公司都有其自身需要面对的挑战,但它们都实现了盈利。
  }
\end{verseparallel}

\begin{table}[!htbp]
  \centering
  \begin{center}
    \begin{tabular}{ccc}
      \toprule
      \makecell[c]{(\$ \textit{millions}) \\ (\textit{百万美元})} & \underline{2015} & \underline{2014} \\
      \textbf{GEICO} & & \\
      \makecell[c]{Premiums earned \\ 保费收入} & 22,718 & 20,496 \\
      \makecell[c]{Underwriting gain/ (loss) -pre-tax \\ 税前承保利润/ (损失)} & 22,718 & 20,496 \\
      \textbf{General Re} & & \\
      \makecell[c]{Premiums earned \\ 保费收入} & 5,975 & 6,264 \\
      \makecell[c]{Underwriting gain/ (loss) -pre-tax \\ 税前承保利润/ (损失)} & 132 & 277 \\
      \textbf{Berkshire Hathaway Reinsurance Group} & & \\
      \makecell[c]{Premiums earned \\ 保费收入} & 7,207 & 10,116 \\
      \makecell[c]{Underwriting gain/ (loss) -pre-tax \\ 税前承保利润/ (损失)} & 132 & 277 \\
      \textbf{Berkshire Hathaway Primary Group} & & \\
      \makecell[c]{Premiums earned \\ 保费收入} & 5,394 & 4,377 \\
      \makecell[c]{Underwriting gain/ (loss) -pre-tax \\ 税前承保利润/ (损失)} & 824 & 626 \\
      \makecell[c]{Total underwriting gain/ (loss) -pre-tax \\ 税前承保总利润/ (损失)} & 1,837 & 2,668 \\
      \makecell[c]{Average float \\ 平均浮存金} & 85,822 & 80,581 \\
      \makecell[c]{Cost of float \\ 浮存金成本} & (2.1\%) & (3.3\%) \\
      \bottomrule
    \end{tabular}
    \caption{Berkshire Hathaway --- Insurance Underwriting \\ 伯克希尔保费、承保
      利润和浮存金数据}
  \end{center}
\end{table}
\end{subsection}

\begin{subsection}{GEICO}
\begin{verseparallel}
  {
    GEICO reported mixed results. In the plus column, a combination of rate
    increases and policyholder growth expanded earned premiums 10.8\% to \$22.7
    billion. Its market share grew from 10.8\% to 11.4\%. Underwriting expenses
    (at 15.9\% of premiums) had a fourth consecutive year of improvement.
    That's where the good news ended. Losses ballooned by 4.4 percentage points
    to 82.1\% of premiums. The cause was an increase in both frequency and
    severity of claims. Such an increase could only be attributed to more
    drivers using smartphones\endnote{123} \endnote{123}. The higher loss
    experience caused GEICO's underwriting profit to decline 60\% to \$460
    million, a 98\% combined ratio. It would have to increase premium rates even
    more to counter the higher loss experience and return to historical rates of
    profitability. \\
  }
  {
    GEICO 这一年喜忧参半。可喜的是,投保费率上升,投保人数量增加,使其保费收入同
    比增长10.8\%,达到 227 亿美元,其市场份额也从 10.8\% 增长至 11.4\%。承保费
    用(占保费的 15.9\%)连续第四年下降。这些是为数不多的好消息了。GEICO 的亏损激
    增4.4\%,达到保费的 82.1\%,原因是索赔的频率更高,同时赔偿标准也更高。这是因
    为越来越多的司机开始使用智能手机。更高的保险赔偿也导致 GEICO 的承保利润同比下
    降 60\%,仅为 4.6亿美元,combined ratio 98\%。GEICO 不得不进一步提高投保费
    率,从而抵消更多的保险赔偿,争取能恢复之前的盈利水平。
  }
\end{verseparallel}
\end{subsection}

\begin{subsection}{General Re}
\begin{verseparallel}
  {
    General Re also faced headwinds in 2015. High industry capacity depressed
    pricing and reduced Gen Re's appetite for new business. But Gen Re remained
    profitable, a reflection of its culture of aiming for underwriting profits
    irrespective of volume. Its overall pre-tax underwriting gain fell 52\% to
    \$132 million on earned premiums that declined 5\% to \$6 billion. Gen Re was
    the only insurance unit to experience a decline in float during the year,
    which fell 3.7\% to \$18.6 billion. \\
  }
  {
    2015年,General Re 的情况也不太好。其承保能力限制了保费价格,也使 General Re
    放弃继续扩张新业务。但 General Re 仍然保持着盈利,这反映出公司追求的是保证承
    保利润而非更大的规模。它的税前承保收入同比下降 52\%,只有 1.32 亿美元,保费也
    同比下降了5\%,只有 60 亿美元。General Re 是 2015 年伯克希尔唯一一个浮存金减
    少的公司,浮存金同比下降 3.7\%,数额为 186 亿美元。
  }
\end{verseparallel}

\begin{verseparallel}
  {
    Earned premiums in property/casualty were \$2.8 billion, a decline of 10\%
    (2\% adjusted for currency). Pre-tax underwriting profits fell 26\% to \$150
    million. An explosion in Tianjin, China costing \$50 million was the only
    major catastrophe loss, but higher loss ratios elsewhere weighed on
    profitability. Property gains totaled \$289 million and benefitted from
    favorable loss development. Casualty losses of \$139 million included
    charges for discount accretion on workers' compensation liabilities and
    deferred charge amortization on retroactive reinsurance contracts, a drag
    that would continue largely independent of year-to-year changes in premium
    volume. Consistent current year losses on casualty business largely stemmed
    from Gen Re's conservative underwriting. In each year since 2009 the
    casualty lines reported favorable development of prior year business. Gen
    Re's troubled history was close enough in the past to remind it that
    continued discipline was required for good results in reinsurance. \\
  }
  {
    财产/意外伤害保费盈利 28亿美元,同比下降10\%(2\% adjusted for)。税前承保利
    润同比下降 26\%,总额为 1.5 亿美元。中国天津发生的爆炸造成了 5,000 万美元的亏
    损,是唯一的巨额灾难赔偿,但其他业务的更高赔付比率也拖累了盈利能力。房地产保
    险盈利总计 2.89 亿美元,受益于有利的亏损发展。1.39 亿美元的伤亡损失包括工人赔偿
    负债的折现费用以及追溯性再保险合同的递延费用摊销,这一拖累将基本上独立于保费
    规模的同比变化而持续下去。今年意外险业务持续亏损的主要原因是 General Re 的承
    保较为保守。自2009年以来,伤亡线每年均录得上年业务的良好发展。General Re 过去
    的坎坷经历足以警醒它自己,那就是再保险业务要想取得良好的业绩,需要在承保时持
    续自律,不可贸然行动。
  }
\end{verseparallel}

\begin{verseparallel}
  {
    The life/health lines reported losses of \$18 million compared to a \$73
    million gain the year before. Earned premiums were flat at \$3.2 billion but
    would have increased 8\% if not for currency headwinds. New business came
    from markets in Canada and Asia. Weakness in its North American long-term
    care business and in individual life caused profitability to decline.
  }
  {
    人寿/健康业务线亏损1,800万美元,而前一年该业务线还盈利 7,300 万美元。保费收入
    为 32 亿美元,与去年持平,但如果汇率和去年持平的话,这个数值本应增长8\%。新业
    务来自加拿大和亚洲市场,其北美长期护理保险和个人寿险的业务疲软导致其盈利能力
    下降。
  }
\end{verseparallel}
\end{subsection}


\begin{subsection}{Berkshire Hathaway Reinsurance Group}
\begin{verseparallel}
  {

    BHRG also faced headwinds from depressed pricing. It too remained
    profitable, though its pre-tax underwriting gain fell 31\% to \$421 million on
    earned premiums that declined 29\% to \$7.2 billion. Ajit Jain earned his
    usual praise from Buffett in the Chairman's letter. It's easy to see why.
    Even in the face of industry headwinds, Jain's group increased float
    4\% to \$44 billion. \\
  }
  {
    BHRG 也面临定价低迷的不利局面。它也仍然保持盈利,尽管它的税前承包收入同比下
    降31\%,只有 4.21亿美元,同时保费收入也同比下降 29\%,总额为 72 亿美元。Ajit
    Jain 在巴菲特给股东的信中赢得了一如既往的赞扬,原因显而易见,在保险行业形势不
    利的情况下,Jain 的团队仍然使浮存金同比增长 4\%,达到 440 亿美元。
  }
\end{verseparallel}

\begin{verseparallel}
  {
    Earned premiums in property/casualty increased 8\% to \$4.4 billion but
    profits fell by 33\% to \$994 million—a stellar result by any measure. A new
    ten-year, 20\% quota-share contract with Insurance Australia Group that
    started on July 1 more than offset declines in earned premiums from property
    catastrophe, property quota-share, and London markets. The only notable
    catastrophe loss was an \$86 million loss from the same explosion in China
    that hurt Gen Re's results. \\
  }
  {
    财产险和意外险的保费收入同比增长 8\%,总额达到 44 亿美元,但利润下降 33\%,总
    额为 9.94 亿美元。无论以何种标准衡量,这都是一个不错的结
    果。BHRG 与 Insurance Australia Group 从 2015 年 7 月 1 日起签订了一份期限为
    十年,持股比例为 20\% 的新合同,这份合同的签署很大程度上抵消了由于房地产灾难、
    房地产份额和伦敦市场等因素造成的保费收入下降,为 BRHG 提供了很大一笔保费收入。
    唯一显眼的重大损失也是发生在中国天津的爆炸,造成了 8600 万美元的损失,这次爆
    炸也给 General Re 带来了很大的损失。
  }
\end{verseparallel}

\begin{verseparallel}
  {
    The retroactive reinsurance segment all but disappeared. Premiums written
    and earned amounted to just \$5 million in 2015, down 99.9\% from the \$3.4
    billion the year before. Such a large decline reflected Berkshire's
    willingness to walk away when business was not available at appropriate
    prices. Shutting off the spigot revealed the impact of the deferred charges
    Buffett frequently pointed to as a drag on reported earnings. The
    retroactive segment reported a loss of \$469 million with all but \$60
    million stemming from deferred charge amortization. \endnote{123} \\
  }
  {
    追溯再保险业务几乎消失。2015年,其净保费收入仅为 500 万美元,较上年 34 亿美元
    下降了 99.9\%。如此大的跌幅反映出伯克希尔公司在价格不利时宁愿放弃这部分业务。
    放弃这部分业务的一个后果是,巴菲特经常提到的递延费用拖累了追溯再保险部分的财
    报利润。追溯再保险业务亏损 4.69 亿美元,其中 6000 万美元来自递延费用摊销。
  }
\end{verseparallel}

\begin{verseparallel}
  {
    Earned premiums from life/health increased 4\% to \$2.8 billion. A loss of
    \$54 million was an improvement from a \$173 million loss the year before.
    In 2015, Berkshire broke down the life/health segment into three additional
    categories. Each category was tied to time-value-of-money concepts and
    produced accounting charges that hid the valuable economics of its float.
    The categories were: \\
  }
  {
    寿险/健康险业务保费收入同比增长4\%,总额 28 亿美元,2015 年亏损 5400 万美元,
    与前一年 1.73 亿美元的亏损相比有所改善。2015年,伯克希尔将寿险/健康险业务细分
    为三类。每个类别都与货币的时间价值相关,因此其会计费用会减少浮存金的经济收益。
    这三类分别是:
  }

\end{verseparallel}

{\color{green}{此处应为列表项 \\}}

\begin{verseparallel}
  {
    \textit{Periodic payment annuity}: Berkshire received upfront premiums and
    made payments stretching over decades. This type of business records
    no gain or loss upfront. Instead, charges are recognized over time like
    the deferred charge amortization on retroactive reinsurance business.
    In this case, they arise because liabilities are discounted upfront to
    account for the time value of money, and the charges (called discount
    accretion) are taken into earnings. \\
  }
  {
    \textit{定期支付年金}:伯克希尔预先收取保费,在其后的数十年间进行赔付或返还。
    这类业务在前期不会记录盈利或亏损。相反,只有随着时间慢慢推移才能确认这笔保险
    是盈利还是亏损,就像追溯再保险业务的递延费用摊销一样。在这种情况下,最开始需
    要对保费收入带来的负债进行贴现,以反映货币的时间价值,而其后每年的费用()称为贴
    现增值)计入利润。
  }
\end{verseparallel}

\begin{verseparallel}
  {
    \textit{Life reinsurance}: Berkshire took the risk from the direct writers of
    life insurance.
  }
  {
    \textit{人寿再保险}:伯克希尔承担了人寿保险直接承保人的风险。
  }
\end{verseparallel}

\begin{verseparallel}
  {
    \textit{Variable annuity}: This business guaranteed closed blocks of
    variable annuity business written by direct writers..
  }
  {
    \textit{可变年金}:这部分与直接承保人的可变年金业务相对应。
  }
\end{verseparallel}

\end{subsection}

\begin{subsection}{Berkshire Hathaway Primary Group}
\begin{verseparallel}
  {

    The Primary Group grew earned premiums 23\% to \$5.4 billion and pre-tax
    underwriting profit by 32\% to \$824 million (an 84.7\% combined ratio).
    Major contributors were the new Berkshire Hathaway Specialty Group, NICO
    Primary, the Home State companies, and GUARD.\@ BH Specialty Group grew
    premium volume to \$1 billion, an incredible achievement considering the unit
    was formed in 2013.

  }
  {
    Primary Group 实现保费收入 54 亿美元,增长23\%,税前承保利润 8.24 亿美元,增
    长32\%(综合比率为84.7\%)。主要贡献者是新成立的 Berkshire Hathaway Specialty
    Group、NICO Primary、Home State companies 和 GUARD。BH Specialty Group 的保费
    收入增长到了 10 亿美元,考虑到该部门是在 2013 年才组建的,这是一个不可思议的
    成就。
  }
\end{verseparallel}

\end{subsection}

\begin{subsection}{Regulated, Capital-Intensive Businesses(强监管重资本行业)}
\begin{verseparallel}
  {

    In 2015 BNSF regained its good graces with Buffett by improving its service
    levels. Just as Buffett predicted the previous year, the railroad's
    financial performance followed suit. Pre-tax profits grew 10\% to a record
    \$6.7 billion. Bolstering growth were massive capital expenditures of \$5.7
    billion—almost three times depreciation charges.\endnote{123} This was as it
    should be. Berkshire's railroad carried 17\% of all intercity freight in the
    US during the year. \\

  }
  {
    2015年,BNSF 通过提高服务水平重新赢得了巴菲特的青睐。它的财务业绩与巴菲特在前
    一年的预测完美契合,其税前利润增长 10\%,达到创纪录的 67 亿美元。而支撑其增长
    的是 57 亿美元的巨额资本支出,几乎是折旧支出的三倍。这也是理所应当的。2015年,
    伯克希尔铁路运输量占美国所有城际货运的 17\%。
  }
\end{verseparallel}

\begin{verseparallel}
  {
    BNSF's 5.5\% decline in revenues to \$22 billion illustrates the importance
    of understanding the underlying business model of a company and paying
    attention to the right variables. One of the largest expenses of any
    railroad is fuel, and most pass these costs through to shippers. A 41\%
    decline in fuel costs during the year was the major reason why BNSF's
    revenues declined during 2015. Its freight volume was flat at 10.3 million
    units, which shows the railroad regaining control of expenses. \\
  }
  {
    BNSF 的营收下降 5.5\% 至 220 亿美元,这也说明理解公司的商业模式,抓住关键变量
    的重要性。铁路行业最大的开销之一就是燃料,而大多数铁路公司将燃料成本转嫁给了
    运输者。2015 年 BNSF 营收下降的主要原因是燃料成本下降了 41\%。该公司的货运量
    为 1030 万单元,与去年持平,表明该公司重新控制住了成本。
  }
\end{verseparallel}

\begin{verseparallel}
  {
    Berkshire Hathaway Energy (BHE) increased EBIT 6.8\% to \$3.4 billion.
    Berkshire's share of net earnings grew 13\% to \$2.1 billion. A large part
    of that increase came from the addition of Alta Link, the Alberta, Canada-
    based electric distribution business acquired in late 2014. BHE's existing
    businesses continued to generate the stability inherent in their business
    models. Two items of note affected the financials. One was a strong increase
    in the value of the US dollar. This had the effect of reducing reported
    revenues and earnings from UK-based Northern Powergrid. The other item
    affecting the financials was a decline in energy costs. Like BNSF, BHE
    passed along these savings to customers. This was to be expected from a
    heavily regulated business. \\
  }
  {
    Berkshire Hathaway Energy 息税前利润为 34 亿美元,同比增长 6.8\%。伯克希尔持
    有股份对应的净利润为 21 亿美元,同比增长 13\%。利润的增长很大一部分来自
    于 Alta Link,该公司是一家总部位于加拿大艾伯塔省的电力分销企业,于 2014 年末
    被 BHE 收购。BHE 之前拥有的企业表现平稳,体现了其商业模式中固有的稳定性。有两
    件事影响了 BHE 的财报。其中之一是美元价值的强劲增长,这导致英国北部电网的营收
    和利润减少。另一件是能源成本的下降。与 BNSF 一样,BHE 将成本下降带来的优惠转
    移给了客户,在一家受到严格监管的企业此举属于意料之中。
  }
\end{verseparallel}
\end{subsection}

\begin{subsection}{Manufacturing, Service and Retailing(制造业、服务业和零售业)}

\begin{verseparallel}
  {

    Berkshire again revised its presentation of the MSR business (nothing
    changed operationally). The businesses were split into two broad categories:
    manufacturing businesses, and service and retailing businesses. Both were
    further delineated into three main segments (see Table 9.7). McLane was
    reported separately because its revenues were large compared to Berkshire's
    total. Earnings for the group totaled \$36.1 billion, down 2\%. Comparative
    results (undisclosed) were poorer considering Berkshire made acquisitions in
    this segment during the year. \\
  }
  {
    伯克希尔再次修改了对 MSR 业务的表述(运营方面没有任何变化)。这些业务被分为两
    大类:制造业以及服务和零售业。每大类又进一步划分为三小类(见表9.7)。之所以单
    独披露 McLane 的营收,是因为其在总营收中比重不小。该业务的营收总额为 361 亿美
    元,同比下降 2\%({\color{red}{这里原文有误,应为营收,而非利润}})。考虑到伯
    克希尔在 2014 年还在这个业务方向进行了收购,未披露的不计入新收购企业的营收就
    更少了。
  }
\end{verseparallel}

\begin{table}[!htbp]
  \centering
  \begin{center}
    \begin{tabular}{cccc}
      \makecell[c]{(\$ \textit{millions}) \\ (\textit{百万美元})} & \underline{2015} & \underline{2014} & \makecell[c]{\%Change \\ 同比增长} \\
      \toprule
      \makecell[c]{Industrial products \\ 工业用品} & 2,994 & 3,159 & (5\%) \\
      \makecell[c]{Industrial products \\ 建筑业用品} & 1,167 & 896 & 30\% \\
      \makecell[c]{Industrial products \\ 消费业用品} & 732 & 756 & (3\%) \\
      \makecell[c]{\textbf{Subtotal --- manufacturing} \\ \textbf{制造业合计}} & 4,893 & 4,811 & 2\% \\
      \midrule
      \makecell[c]{Service \\ 服务业} & 1,156 & 1,202 & (4\%) \\
      \makecell[c]{Retailing \\ 零售业} & 564 & 344 & 64\% \\
      \makecell[c]{McLane \\ McLane} & 502 & 435 & 15\% \\
      \makecell[c]{\textbf{Subtotal --- service and retailing} \\ \textbf{服务和零售业合计}} & 2,222 & 1,981 & 12\% \\
      \midrule
      \makecell[c]{Total pre-tax earnings \\ 税前利润总计} & 7,115 & 6,792 & 5\% \\
      \makecell[c]{Income taxes and non-controlling interests \\ 所得税和少数股东权益} & (2,432) & (2,324) & 5\% \\
      \makecell[c]{Earning after tax \\ 税后净利润} & 4,683 & 4,468 & 5\% \\
      \bottomrule
    \end{tabular}
    \caption{Manufacturing, Service, and Retailing businesses --- pre-tax
      earnings \\ 制造业、服务业和零售业税前利润}
  \end{center}
\end{table}

\begin{verseparallel}
  {
    With so many businesses to report, the categories were logical. But some
    analysts pined for additional data. Results from large companies like Shaw,
    Lubrizol, IMC (as the parent company of Iscar \endnote{123}), and Marmon
    were aggregated and discussion squeezed into just a few paragraphs along
    with many other businesses instead of being individually reported. Some had
    previously been public companies that produced annual reports hundreds of
    pages long. Much of that data was now gone as part of Berkshire’s
    reporting. Many of the businesses were similar enough that the consolidated
    data was still valuable. The report did identify specific businesses where
    the impact was meaningful, but Berkshire’s growth diminished the importance
    of individual businesses compared to the whole. \\
  }
  {
    由于这部分业务的公司实在太多,因此将它们分类汇总呈现在报表中是合乎逻辑的。但
    一些分析师渴望获得更多的数据。像 Shaw、Lubrizol、IMC(Iscar 的母公
    司) Marmon 这样的大公司的业务情况被压缩到几段文字,和许多其他业务的数据合并
    在一起而非单独呈现。这些公司里有些以前是上市公司,每年的年报长达数百页。如今,
    这些公司的数据有很大一部分不出现在伯克希尔的年报之中。由于这部分业务的许多公
    司非常相似,因此整合后的数据仍然有它的价值。虽然年报中明确说明哪些业务对伯克
    希尔有重大影响,但伯克希尔的增长削减了该业务线中单个公司的重要程度。
  }
\end{verseparallel}

\begin{verseparallel}
  {
    \textbf{Industrial Products (revenues of \$16.8bn, down 5\%)}: A big part of the 5\%
    decline in pre-tax to \$3 billion earnings from industrial products came
    from a stronger US dollar. IMC was likely responsible for most of the
    impact, as it was the largest business in the segment and located overseas.
    A slowdown in demand began over the second half of the year and was expected
    to continue into 2016. \\
  }
  {
    \textbf{工业用品(收入 168 亿美元,同比下降 5\%)}:工业产品税前利润为 30 亿
    美元,同比下降 5\%,其中很大一部分原因来自美元走强。IMC 可能是受影响最大的公
    司,因为它是该细分业务线的最大业务,同时位于美国之外。这部分业务的需求放缓始
    于 2015 年下半年,预计将持续到2016年。
  }
\end{verseparallel}

\begin{verseparallel}
  {
    \textbf{Building Products (revenues of \$10.3bn, up 1.9\%)}: This was the
    only manufacturing segment to increase earnings. The large 30\% jump in
    earnings to \$1.2 billion on just a 2\% increase in revenues was a result of
    higher unit volume, lower raw materials costs, and energy savings, offset by
    the strong US dollar and restructuring costs. Bolt-on acquisitions increased
    earnings as well. The large increase in earnings illustrated the pricing
    power of those businesses. The segment included Shaw, Johns Manville, Acme
    Building Brands, Benjamin Moore, and MiTek. These businesses did not face
    price regulation and were not required (like Berkshire Hathaway Energy and
    BNSF) to pass along savings to customers. Crucially, their competitive
    positions did not require it either. Some unregulated businesses nonetheless
    face competition so intense they must pass on savings to customers to retain
    business. \\
  }
  {
    \textbf{建筑业用品(收入103亿美元,增长1.9\%)}:这是唯一一个盈利增长的制造业业
    务。营收仅增长2\%,但利润大幅增长30\%,达到 12 亿美元,这是因为单位销量增加、
    原材料成本降低以及能源上的节约,但一部分利润被美元走强和重组带来的成本所抵消。
    补强收购也对盈利的提高有所贡献。盈利的大幅增长展现了这些企业的定价能力。该细
    分业务包括 Shaw、Johns Manville、Acme Building Brands、Benjamin
    Moore 和 MiTek。这些公司没有受到价格管制,也不需要(像 BHE 和 BNSF 那样)将成
    本的节约让给客户。最关键的是,它们在竞争中处于优势,因此也不需要这样做。尽管
    如此,该业务线中一些不受监管的公司仍然面临着相当激烈的竞争,它们必须将节约的
    成本让给客户,才能保证公司能继续运营。
  }
\end{verseparallel}

\begin{verseparallel}
  {
    \textbf{Consumer Products (revenues of \$9.1 bn, flat)}: Earnings from
    consumer products fell 3\% to \$732 million because of a loss at Fruit of
    the Loom (related to selling an unprofitable unit) and declines in footwear.
    Earnings at Forest River increased on higher unit sales and increased
    prices. \\
  }
  {
    \textbf{消费业用品(收入为91亿美元,持平)}:消费业用品利润下降3\%,总额
    为 7.32 亿美元,原因是 Fruit of the Loom(出售了一个无利可图的业务)的亏损以
    及鞋类产量的下滑。由于销售额上升和价格上涨,Forest River 的利润有所增加。
  }
\end{verseparallel}

\begin{verseparallel}
  {
    \textbf{Service (revenues of \$10.2bn, up 3.5\%)}: This was the only service
    and retailing segment to see a decline in earnings, which fell 4\% to \$1.2
    billion. NetJets expanded operations but faced lower margins and higher
    costs (including a one-time lump-sum payment related to a collective
    bargaining agreement) that weighed on the results of the entire segment.
    Newspaper revenues (and presumably profits) declined. Offsetting this was
    the addition of WPLG, the Miami, Florida television station Berkshire
    acquired in 2014, and the addition of Charter Brokerage. \\
  }
  {
    \textbf{服务行业(收入102亿美元,增长3.5\%)}:这是服务和零售业务中唯一一个盈
    利下降的细分业务,利润下降 4\%,总额为 12 亿美元。NetJets 的业务进行了扩张,
    但面临利润率下降和成本上升(包括与集体谈判协议相关的一次性一次性付款)的问题,
    这拖累了整个细分业务的业绩。报纸收入(以及它未披露的利润可能)下降。与下降相
    反的是,伯克希尔在 2014 年收购的 WPLG,the Miami,Florida 电视台,以及对
    Charter Brokerage 的收购。
  }
\end{verseparallel}

\begin{verseparallel}
  {
    \textbf{Retailing (revenues of \$13.3bn, up 214\%)}: This segment welcomed
    two new businesses in 2015 which increased pre-tax earnings by 64\% to \$564
    million. The first was the Van Tuyl Group, a group of eighty-one automotive
    dealerships located in ten (mostly western US) states. The acquisition also
    included Van Tuyl’s two related insurance businesses, two auto auctions,
    and a distributor of automotive fluid maintenance products. Buffett met
    Larry Van Tuyl years before and the Van Tuyls decided Berkshire would be a
    good permanent home for the business. Upon joining Berkshire Van Tuyl was
    renamed Berkshire Hathaway Automotive. The business was built by Larry Van
    Tuyl and his father, Cecil, over sixty-two years. A key insight the Van
    Tuyls had, and Buffett shared, was creating a sense of ownership with each
    local manager. ``We will continue to operate with extreme—indeed, almost
    unheard of—decentralization at Berkshire,'' he explained to shareholders.
    This allowed them to successfully grow the business to the fifth-largest
    auto group in the US. \endnote{123} \\
  }
  {
    \textbf{零售业(收入 133 亿美元,增长 214\%)}:这一细分业务在 2015 年迎来了
    两家新公司,税前利润增长 64\%,达到 5.64 亿美元。第一家是 Van Tuyl Group,这
    是一个由 81 家汽车经销商组成的集团公司,分布在 10 个州(主要位于美国西部)。
    此次收购还包括 Van Tuyl 两个相关的保险业务、两次汽车拍卖以及一家汽车流体维护
    经销商。早在多年前,巴菲特就认识了 Larry Van Tury,范·图尔夫妇认为伯克希尔可
    以为自己的企业提供一个永远的家。加入伯克希尔后,Van Tuyl 更名为Berkshire
    Hathaway Automotive。Larry Van Tury 和他的父亲 Cecil 在 62 年的时间里建立了这
    家公司。Van Tuyl 夫妇的核心观点,也是巴菲特和分享给大众的观点,就是要让每一位
    子公司的基金经理都有归属感。“我们将继续在伯克希尔进行极端——事实上,几乎闻所
    未闻——力度的放权,”他向股东解释道。正是这种放权使得 Van Tuyl 能够成功地成长
    为美国第五大汽车集团。
  }
\end{verseparallel}

\begin{verseparallel}
  {
    A limited amount of data was available on the Van Tuyl acquisition. Buffett
    put the company’s annual sales volume at \$9 billion. Industry commentators
    thought pre-tax earnings might be between \$350 and \$471 million.
    \endnote{123} The purchase price of \$4.1 billion included \$1.3 billion in
    cash and investments. Adjusting for cash, it appears Berkshire paid between
    six- and eight-times pre-tax earnings. This apparent bargain price looks
    less rich considering that light vehicle sales (sales of cars, vans, SUVs
    and smaller pickup trucks) were at or near recent highs of about 17 million
    annually. \endnote{123} \\
  }
  {
    收购 Van Tuyl 的具体数据并不多。巴菲特估算这家公司的年销售额为 90 亿美元。业
    内评论人士则认为,该公司的税前利润可能在 3.5 亿至 4.71 亿美元之间。伯克希尔的
    收购价为 41 亿美元,包括 13 亿美元的现金和投资部分。现金调整后,伯克希尔收购
    的价格大约是税前利润的 6 到 8 倍。考虑到小汽车的销量(轿车、面包车、SUV和小型
    皮卡的销量)在每年 1700 万辆的近期高点,这个看起来明显便宜的价格显得也没那么
    划算。
  }
\end{verseparallel}

\begin{verseparallel}
  {
    The second acquisition of 2015 in this segment was Detlev Louis Motorrad.
    The company was one of the largest retailers of motorcycle accessories in
    Germany. The acquisition was too small to be detailed in the Berkshire
    Annual Report. Some sources put its annual revenues at around 270 million
    Euros (about \$300 million) and the purchase price at about 400 million
    Euros (about \$444 million). \endnote{123} The deal was notable in another way.
    Buffett tapped Ted Weschler to negotiate the deal and then made him chairman
    to oversee the investment. \\
  }
  {
    2015 年伯克希尔在该细分业务的第二次收购是对 Detlev Louis Motorrad 的收购。该
    公司是德国最大的摩托车配件零售商之一。由于此次收购规模太小,在伯克希尔的年报
    中没有详细的信息。一些消息人士估计,该公司的年收入约为 2.7 亿欧元(约3亿美
    元),收购价约为 4 亿欧元(约 4.44 亿美元)。 \endnote{1} 这笔收购还有另一个值
    得注意的地方,巴菲特选了 Ted Weschler 负责收购交易的谈判,然后任命他为董事长
    继续负责这笔投资。
  }
\end{verseparallel}

\begin{verseparallel}
  {
    Earnings from Berkshire Hathaway Automotive and Detlev Louis Motorrad were
    the primary reason for the 64\% increase in retailing earnings in 2015.
    Furniture retailing revenues increased 24\% from the new Nebraska Furniture
    Mart store in Texas and increases from RC Willey and Jordan’s. \\
  }
  {

    Berkshire Hathaway Automotive 和 Detlev Louis Motorrad 的盈利是 2015 年伯克希
    尔零售业利润增长 64\% 的主要原因。来自德克萨斯州的 Nebraska Furniture Mart 新
    店开业,RC Willey and Jordan’s 营收增长,使伯克希尔家具零售收入增长了 24\%。

  }
\end{verseparallel}

\begin{verseparallel}
  {
    \textbf{McLane (revenues of \$48.2bn, up 3\%)}: Berkshire’s only standalone
    business other than BNSF, McLane, increased volumes in foodservice (up
    6\%), beverage (up 8\%), and grocery (up 2\%). This was another business
    able to directly benefit (at least in the short run) from the decline in fuel
    costs. Pre-tax earnings grew 15\% to \$502 million. A \$19 million gain (or
    about 4 percentage points) came from a one-time gain from the sale of an
    undisclosed subsidiary.
  }
  {
    \textbf{McLane(营收 482亿美元,增长3\%)}:McLane 作为伯克希尔除 BNSF 外唯一
    的独立公司,在食品、饮料和杂货上都有增长(分别增长 6\%、8\% 和 2\%)。这是另
    一家能够直接受益于(至少在短期内)成本下降的企业。其税前利润增长 15\%,达
    到 5.02 亿美元。其中 1900 万美元的收益(约 4\%)来自一次性地出售一家未披露子
    公司。
  }
\end{verseparallel}
\end{subsection}

\begin{subsection}{Finance and Financial Products(金融和金融产品)}
\begin{verseparallel}
  {
    Pre-tax earnings in Finance and Financial Products jumped 13\% to \$2.1
    billion. The major driver of the increase was a 27\% increase (to \$706
    million) in earnings from Clayton. Clayton increased unit sales and
    benefitted from lower interest costs and lower delinquencies/foreclosures.
    Marmon’s tank leasing business (UTLX) and XTRA’s trailer leasing business
    were lumped into transportation equipment leasing. Pre-tax earnings of those
    businesses increased 10\% to \$909 million. Part of that increase was a \$1
    billion purchase of 25,085 tank cars from General Electric (bringing its
    total to 133,280). UTLX also acquired several businesses during the year to
    continue building out its full-service maintenance operation. Berkshire
    couldn’t come close to the financing advantage of banks to conduct pure
    leasing operations. Both the tank leasing business and XTRA’s trailer
    leasing business had important service components that added value above and
    beyond a simple financing arrangement. Everything else from CORT to Berkadia
    and the fees charged to Clayton and NetJets for use of Berkshire’s credit
    fell to the other category. Earnings in that category increased 4\% to \$471
    million. \\
  }
  {

    金融和金融产品公司的税前利润增长 13\%,达到 21 亿美元,这主要归功于 Clayton利
    润增长 27\% 总额达到 7.06 亿美元。克莱顿的单位销售额上升,同时受益于较低的利
    息成本和较低的房贷拖欠和强制法拍。Marmon 的油罐车租赁业务(UTLX) 和 XTRA 的
    拖车租赁业务合并为运输设备租赁。这些公司的税前利润增长 10\%,达到 9.09 亿美元,
    其中一部分来自通用电气 10 亿美元的 25,085 辆油罐车采购订单(采购后其油罐车总
    数达到 133,280 辆)。年内,UTLX 亦收购多项业务,以继续建设其全方位维修业务。在
    纯粹的租赁业务方面,伯克希尔无法和银行的融资优势相媲美。油罐车租赁业务
    和 XTRA 的拖车租赁业务都包含重要的服务部分,这些服务除了简单的融资安排之外,
    还增加了价值。除此之外,CORT、Berkadia 以及向 Clayton 和 NetJets 收取的信用额
    度的费用,所有这些都归入了另一个分类。该分类的收入增长4\%,达到4.71亿美元。

  }
\end{verseparallel}
\end{subsection}

\begin{subsection}{Investments(投资)}
\begin{verseparallel}
  {

    Berkshire's investment portfolio saw few changes in 2015. A net \$1.5
    billion was invested in equities funded by a few sales. The most notable
    sales were Berkshire’s positions in Swiss Re and Munich Re, two
    European-based reinsurers. The position in Munich was valued at \$4 billion
    at year-end 2014; Swiss Re was too small to be specifically identified.
    Buffett elaborated on his reasoning for selling the investments at the 2016
    Annual Meeting. He said it came down to two factors, and neither was
    related to management, which he continued to admire:    

  }
  {
    2015 年,伯克希尔的投资组合头寸几乎没有变化。新买入 15 亿美元的股票,资金是通
    过几次卖出之前投资的股票所得到的。最引人注目的是伯克希尔减持 Swiss
    Re 和 Munich Re 这两家欧洲再保险公司的股票。2014年底,伯克希尔投资 Munich Re
    的股票价值为 40 亿美元;投资 Swiss Re 的规模太小,没办法确定具体数额。巴菲特
    在 2016 年年会上详细阐述了减持的理由。他说,这归结于两个因素,但两者都与管理
    层无关,他仍然很欣赏这两家公司的管理层:
  }
\end{verseparallel}    

{\color{green}{此处应为列表项 \\}}

\begin{verseparallel}
  {
    1. An influx of capital that flooded the reinsurance industry and pressured
    premium rates. This was likely to continue for some time. \\
  }
  {
    1. 大量资本涌入再保险行业,给保险费率带来压力。这种情况可能还会持续一段时间。
  }
\end{verseparallel}

\begin{verseparallel}
  {
    2. Low interest rates made insurance float much less valuable. That was more
    important for European-based insurers since interest rates there were low or
    even negative. Berkshire would be hurt by the competition and low rates, but
    it had more options to invest its float, including the acquisition of
    non-insurance subsidiaries. \\
  }
  {
    2. 低利率使得保险浮存金的价值大大降低。这对总部位于欧洲的保险公司影响更大,因
    为那里的利率很低,甚至是负值。竞争和低利率可能会对伯克希尔造成伤害,但它有更
    多的选择来使用浮存金进行投资,包括收购非保险行业的公司。
  }
\end{verseparallel}

\end{subsection}

\begin{subsection}{Productivity and Prosperity(生产力与繁荣)}
\begin{verseparallel}
  {
    Productivity is the amount of output per hour of labor input. Buffett
    devoted a section of the 2015 Chairman's letter to productivity, connecting
    it to both Berkshire's and America's prosperity. The topic was timely (and
    probably prompted by) Heinz and Kraft, whose new management at 3G Capital
    were known for ruthlessly improving productivity, most often by reducing
    headcounts. Buffett thought the connection between productivity and
    prosperity was not entirely clear to some, so he provided examples. \\
  }
  {
    生产力是每小时劳动投入的工作产出。巴菲特在 2015 年致股东信中专门用了一节来探
    讨生产力问题,把它与伯克希尔和美国的繁荣联系起来。这个话题对于 Heinz and
    Kraft 来说非常及时(很可能就是这两家公司提出来的),因为他们的新管理者 3G
    Capital 以无情地裁员以提高生产力而著称。巴菲特认为,很多人不明白生产力和繁荣
    之间的联系,所以他举了一些例子。
  }
\end{verseparallel}

{\color{green}{此处应为列表项 \\}}

\begin{verseparallel}
  {
    \textit{Farming}: The most dramatic example was America's shift away
    from farming during the 20th century. In 1900, 40\% of the country
    was employed growing America's food. As of 2015, just 2\% of the
    population worked on farms. Productivity allowed this to happen,
    beginning with the invention and perfection of the tractor and
    extending to better farming techniques and seed quality. \\
  }
  {
    \textit{农业}:最戏剧性的例子是美国 20 世纪农业的变迁。1900 年美国 40\% 的人
    口都在从事农业,而到了 2015 年,这个数字变成了 2\%。拖拉机的发明和性能完善,
    更高级的农业技术和种子质量等带来了生产力的提高,从而大大减少了从业人口数量。
  }
\end{verseparallel}

\begin{verseparallel}
  {
    \textit{Railroading}: After World War II there were 1.35 million workers
    employed in the railroad industry, and they moved 655 billion revenue
    ton-miles. Fast forward to 2014 and Class I railroads moved 1.85 trillion
    ton-miles with just 187,000 workers. The result was a 55\% decline in the
    inflation-adjusted cost of moving a ton-mile of freight. Safety improved
    dramatically too. Using BNSF as an example, Buffett said injuries fell 50\%
    from 1996. \\
  }
  {
    \textit{铁路}:二战后,铁路行业雇用了 135 万名工人,他们带来了 6550 亿吨-英里
    的收入。而到了 2014 年,I 类铁路收入为 1.85 万亿吨-英里,只用了 187,000 名工
    人。其结果是,每吨-英里运费在经过通胀调整后的成本下降了 55\%,安全性也显著提
    高。以 BNSF 为例,巴菲特指出其受伤人数比 1996 年下降了 50\%。
  }
\end{verseparallel}

\begin{verseparallel}
  {
    \textit{Utilities}: Berkshire Hathaway Energy's (BHE) Iowa utility in 1999
    employed 3,700 people and produced 19 million megawatt-hours of electricity.
    Fast forward to 2015 and it generated 29 million megawatt-hours while
    employing just 3,500 people. Such improvements in productivity allowed BHE
    to keep rates the same for sixteen years. Like BNSF, safety improved too. \\
  }
  {
    \textit{公用事业}:1999年,BHE 爱荷华州公司雇用了3700 人,发电量为 190 亿度。
    而到了 2015 年,发电量为 290 亿度,员工数量降低到 3500 人。生产率的提高使
    得 BHE 在 16 年维持了一定的增长。与 BNSF 一样,安全性也有所提高。
  }
\end{verseparallel}

\begin{verseparallel}
  {
    The examples above proved that increased productivity resulted in real gains
    to civilization and allowed more to be employed in other industries. But
    they came with short-term costs, most notably the workers who lost their
    jobs. Buffett was aware of these costs and had experienced some up close.
    When Berkshire shuttered its mills in the mid-1980s (and at Dexter Shoe
    years later) it employed older workers with non-transferrable skills.
    Buffett thought the solution was social safety nets that cushioned the blow
    to the unfortunate workers while leaving productivity to continue working
    its magic for the benefit of society at large. \endnote{123} Both Buffett and Munger
    were clearly on the side of making operations at Kraft Heinz more efficient.
    They detested sloppy operations and were ever on the lookout for
    inefficiencies at Berkshire, noting that once costs crept in, they tended to
    proliferate. A large and highly profitable conglomerate required continual
    diligence to protect itself from such tendencies. \\
  }
  {
    上述例子表明,生产力的提高促进了文明发展,同时可以让更多的人从事新的行业。但
    它也带来了阵痛,尤其是那些失业工人。巴菲特意识到了这些,也经历了一些类似的事
    情。20 世纪 80 年代中期,伯克希尔关闭了几家钢铁厂(几年后德克斯特制鞋公司也关
    闭了),这些企业雇佣的都是年纪较大、技术没有其他用武之地的工人。巴菲特认为,
    解决之道在于建立完善的社会保险,从而减少那些不幸失业劳动者受到的损失,同时让
    生产力继续发挥作用,大大提升整个社会的效益。显然,巴菲特和芒格都支持提
    高 Kraft Heinz 的经营效率。他们讨厌马马虎虎的运营,同时也一直在关注伯克希尔效
    率低下的问题。他们指出,一旦出现成本上升的迹象,很快就会演变为成本激增。伯克
    希尔这样的大型高盈利集团公司,需要持续审慎地关注成本,从而避免成本激增带来的
    影响。
  }
\end{verseparallel}

\end{subsection}

\end{section}

\theendnotes{}

\end{chapter}

%%% Local Variables:
%%% TeX-master: "../master"
%%% End:

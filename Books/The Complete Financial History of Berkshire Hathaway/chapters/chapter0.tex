\chapter{伯克希尔的第一个五十年: 1965--2014}

\texttt{获得非凡的结果,并不一定要做非凡的事情。}

\begin{verseparallel}
  {
    \noindent Extraordinary is the only word that singularly captures the arc of
    Berkshire Hathaway’s fifty-year transformation under Warren Buffett’s
    control. The company that existed at the end of 2014 looked nothing like it
    did fifty years earlier despite bearing the same name. The struggling
    textile company that once formed the foundation of Berkshire Hathaway had
    cracked, leading to major structural issues that once rebuilt became a
    well-respected conglomerate. Its extraordinary transformation took place
    using what in hindsight were fairy ordinary and timeless basic principles of
    business. Applied step by step, year by year, and decade by decade, the
    ordinary was molded into the extraordinary. \\
  }
  {
    如果用一个词描绘在沃伦·巴菲特执掌下的伯克希尔·哈撒韦公司 50 年的转型发展脉络,
    那这个词非“非凡”莫属。2014 年底的这家公司,尽管公司名称没有变,但和 50 年前
    的伯克希尔·哈撒韦公司相比已然脱胎换骨。曾经作为伯克希尔·哈撒韦公司基业的纺织
    公司倒下了,但重建后,成为了广受尊敬的大型集团公司。从后视镜的视角来看,伯克
    希尔·哈撒韦公司的非凡转型过程遵循的其实是极为普通和永恒的基本商业原则。只是
    单纯一天天、一年年地应用这些原则,伯克希尔·哈撒韦公司从平凡被塑造成为不平凡。
  }
\end{verseparallel}

%%% Local Variables:
%%% TeX-master: "../master"
%%% End:

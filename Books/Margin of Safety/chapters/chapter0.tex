\chapter{引言}

\begin{verseparallel}
  {
    \noindent Investors adopt many different approaches that offer little or
    no real prospect of long-term success and considerable chance of
    substantial economic loss. Many are not coherent investment programs at
    all but instead resemble speculation or outright gambling. Investors are
    frequently lured by the prospect of quick and easy gain and fall victim to
    the many fads of Wall Street. My goals in writing this book are twofold.
    In the first section I identify many of the pitfalls that face investors.
    By highlighting where so many go wrong, I hope to help investors learn to
    avoid these losing strategies. \\
  }
  {
    投资者的流派很多,这些流派长期来看几乎不可能成功,甚至很可能给投资者带来严重
    的损失。这些流派有很多并不是严谨的投资方法,更像是投机甚至赌博。投资者经常被
    简单暴富的想法所诱惑,成为华尔街的受害者。我写这本书的目的有两个。在书的第一
    部分,我指出了投资者存在的各种问题。通过分析投资者常犯的错误,我希望能够帮助
    他们放弃失败的投资策略。
  }
\end{verseparallel}

\begin{verseparallel}
  {
    For the remainder of the book I recommend one particular path for investors
    to follow—a value-investment philosophy. Value investing, the strategy of
    investing in securities trading at an appreciable discount from underlying
    value, has a long history of delivering excellent investment results with
    very limited downside risk. This book explains the philosophy of value
    investing and, perhaps more importantly, the logic behind it in an attempt
    to demonstrate why it succeeds while other approaches fail. \\
  }
  {
    在本书的后半部分,我向投资者推荐了一种具体的投资策略——价值投资。价值投资是在
    公司内在价值基础上,再打上很有吸引力的折扣,用这个价格买入公司的股票。过去的
    几十年里,价值投资在十分有限的风险下实现了出色的投资回报。本书介绍了价值投资
    的基本原理,同时更为重要的,介绍了价值投资的内在逻辑,这样才能试图说明为何它
    能在众多失败的投资策略中脱颖而出。
  }
\end{verseparallel}

\begin{verseparallel}
  {
    I have chosen to begin this book, not with a discussion of what value
    investors do right, but with an assessment of where other investors go
    wrong, for many more investors lose their way along the road to investment
    success than reach their destination. It is easy to stray but a continuous
    effort to remain disciplined. Avoiding where others go wrong is an important
    step in achieving investment success. In fact, it almost ensures it. \\
  }
  {
    在本书开头,我决定先介绍投资者做错了什么而非做对了什么,因为大多数投资者在追
    求投资目标的途中迷失了自我。这是因为迷失自我很容易,而坚持投资纪律则需要付出
    很多努力。避免其他投资者所犯的错误是获得良好投资收益的关键一步。实际上,只要
    避免这些错误,几乎就能保证你获得良好的投资收益。
  }
\end{verseparallel}

\begin{verseparallel}
  {
    You may be wondering, as several of my friends have, why I would write a
    book that could encourage more people to become value investors. Don't I run
    the risk of encouraging increased competition, thereby reducing my own
    investment returns? Perhaps, but I do not believe this will happen. For one
    thing, value investing is not being discussed here for the first time. While
    I have tried to build the case for it somewhat differently from my
    predecessors and while my precise philosophy may vary from that of other
    value investors, a number of these views have been expressed before, notably
    by Benjamin Graham and David Dodd, who more than fifty years ago wrote
    Security Analysis, regarded by many as the bible of value investing. That
    single work has illuminated the way for generations of value investors. More
    recently Graham wrote The Intelligent Investor, a less academic description
    of the value-investment process. Warren Buffett, the chairman of Berkshire
    Hathaway, Inc., and a student of Graham, is regarded as today's most
    successful value investor. He has written countless articles and shareholder
    and partnership letters that together articulate his value-investment
    philosophy coherently and brilliantly. Investors who have failed to heed
    such wise counsel are unlikely to listen to me. \\
  }
  {
    就像我的朋友好奇的,你可能也会想,为什么我要写这样一本书,让更多的人变为价值
    投资者。这样难道不会增加竞争,降低我的投资回报吗?这种情况确实有可能出现,但
    我觉得不会出现。价值投资不是一个新生事物。尽管我在本书中试图从和其他人不同的
    角度阐述价值投资,但其实在本书之前已经有很多价值投资的材料。例如五十多年前本
    杰明·格雷厄姆和大卫·多德写的《证券分析》一书,这本书被看做价值投资者的圣经,
    为价值投资者指明了方向。最近,格雷厄姆又写了《聪明的投资者》一书,这本书更通
    俗易懂一些。沃伦·巴菲特,伯克希尔·哈撒韦公司的主席,同时也是格雷厄姆的学生,
    是当今世上最成功的的价值投资者。他写了无数关于价值投资的文章,还有给伯克希尔·哈
    撒韦公司股东的信,这些材料都展示了他聪慧的价值投资理念。如果投资者听不进这些
    先贤的话,自然也不大可能听我的。
  }
\end{verseparallel}

\begin{verseparallel}
  {
    The truth is, I am pained by the disastrous investment results experienced by
    great numbers of unsophisticated or undisciplined investors. If I can persuade
    just a few of them to avoid dangerous investment strategies and adopt sound ones
    that are designed to preserve and maintain their hard-earned capital, I will be
    satisfied. If I should have a wider influence on investor behavior, then I would
    gladly pay the price of a modest diminution in my own investment returns. \\
  }
  {
    实际上,我对那些新手和不守纪律的投资者遭受的巨大损失感到痛心。如果我能说服其
    中的一小部分人放弃危险的投资策略,转向更合理的,同时也能保住他们辛苦挣来的钱
    的投资策略的话,我会觉得很满足。如果我能在投资者行为上有更大的影响力,那么即
    使我个人的投资回报受到一些微小的影响,我也会乐见其成。
  }
\end{verseparallel}

\begin{verseparallel}
  {
    In any event this book alone will not turn anyone into a successful value
    investor. Value investing requires a great deal of hard work, unusually
    strict discipline, and a long-term investment horizon. Few are willing and
    able to devote sufficient time and effort to become value investors, and
    only a fraction of those have the proper mind-set to succeed. \\
  }
  {
    仅靠这本书的话,任何人都不会成为成功的价值投资者。价值投资需要付出艰辛的努力,
    要严格遵守投资纪律,以及需要有长时间的投资经验。很少有人愿意花如此多的时间和
    精力成为一名价值投资者,而价值投资者中也只有一部分人拥有取得成功的头脑。
  }
\end{verseparallel}

\begin{verseparallel}
  {
    This book most certainly does not provide a surefire formula for investment
    success. There is, of course, no such formula. Rather this book is a
    blueprint that, if carefully followed, offers a good possibility of
    investment success with limited risk. I believe this is as much as investors
    can reasonably hope for. \\
  }
  {
    这本书当然不会提供一个万无一失的成功投资公式。当然,这种公式其实也不存在。这
    本书会给你提供了一个蓝图,如果你认真遵循的话,大概率会获得投资成功,同时只需
    要承担很有限的风险。我觉得这应该是大多数投资者的合理预期。
  }
\end{verseparallel}

\begin{verseparallel}
  {
    Ideally this will be considered, not a book about investing, but a book
    about thinking about investing. Like most eighth-grade algebra students,
    some investors memorize a few formulas or rules and superficially appear
    competent but do not really understand what they are doing. To achieve
    long-term success over many financial market and economic cycles, observing
    a few rules is not enough. Too many things change too quickly in the
    investment world for that approach to succeed. It is necessary instead to
    understand the rationale behind the rules in order to appreciate why they
    work when they do and don't when they don't. I could simply assert that
    value investing works, but I hope to show you why it works and why most
    other approaches do not. \\
  }
  {

    理想情况下,这本书不仅仅关注投资,也关注如何思考投资。就像很多 8 年级的学生一
    样,一些投资者只记住了公式和规则,表面上看起来做得还不错,但根本不理解他们到
    底在做什么。如果想撑过多轮经济周期的考验取得成功,只观察一些规律是远远不够的。
    投资世界中的一切变化得太快了,只依靠静态的规律很难成功。因此理解规律背后的原
    理,可以帮助我们更好地了解规律什么时候能起作用,什么时候不能。我可以直截了当
    地说,价值投资能帮你取得投资成功,但我也希望能向你证明为什么它能帮你取得成功,
    以及为什么其他很多方法做不到这一点。

  }
\end{verseparallel}

\begin{verseparallel}
  {
    If interplanetary visitors landed on Earth and examined the workings of our
    financial markets and the behavior of financial-market participants, they
    would no doubt question the intelligence of the planet's inhabitants. Wall
    Street, the financial marketplace where capital is allocated worldwide, is
    in many ways just a gigantic casino. The recipient of up-front fees on every
    transaction, Wall Street clearly is more concerned with the volume of
    activity than its economic utility. Pension and endowment funds responsible
    for the security and enhancement of long-term retirement, educational, and
    philanthropic resources employ investment managers who frenetically trade
    long-term securities on a very short-term basis, each trying to outguess and
    consequently outperform others doing the same thing. In addition, hundreds
    of billions of dollars are invested in virtual or complete ignorance of
    underlying business fundamentals, often using indexing strategies designed
    to avoid significant under performance at the cost of assured mediocrity. \\
  }
  {
    如果外星人来到地球,看到我们的金融市场,以及金融市场参与者的行为,他们一定会
    质疑我们的智商。华尔街,作为全球财富分配的金融市场,在很多方面就像一个大赌场。
    华尔街靠对每笔交易征收手续费赚钱,所以他们更关心交易的多少而非是否对经济有帮
    助。退休金、养老金、慈善基金等雇佣的基金经理,将这些本来用于改善长期生活的钱,
    狂热频繁地交易本应长期持有的股票。每个基金经理都想着能超过其他人,而其他人也
    都是这么想的。此外,几千亿美元的资金不考虑公司基本面,直接采用指数策略,这样
    能避免长期业绩不佳,但代价是只能获得平均收益。
  }
\end{verseparallel}

\begin{verseparallel}
  {
    Individual and institutional investors alike frequently demonstrate an
    inability to make long-term investment decisions based on business
    fundamentals. There are a number of reasons for this: among them the
    performance pressures faced by institutional investors, the compensation
    structure of Wall Street, and the frenzied atmosphere of the financial
    markets. As a result, investors, particularly institutional investors,
    become enmeshed in a short-term relative performance derby, whereby
    temporary price fluctuations become the dominant focus.
    Relative performance-oriented investors, already focused on short-term
    returns, frequently are attracted to the latest market fads as a source of
    superior relative performance. The temptation of making a fast buck is
    great, and many investors find it difficult to fight the crowd. \\
  }
  {
    个人投资者和机构投资者在做长期投资决策时,很少考虑公司的基本面。忽视基本面的
    原因有很多,比如机构投资者面临的业绩压力、华尔街的报酬结构、金融市场的狂热氛
    围等。上述原因都导致投资者,尤其是机构投资者,被卷入由股价波动主导的短期业绩
    竞赛。追逐相对业绩的投资者主要关注短期回报,会被近期市场上的最新消息吸引。虽
    然短期暴富的诱惑很大,但大多数投资者发现自己很难战胜市场。
  }
\end{verseparallel}

\begin{verseparallel}
  {
    Investors are sometimes their own worst enemies. When prices are generally
    rising, for example, greed leads investors to speculate, to make
    substantial, high-risk bets based upon optimistic predictions, and to focus
    on return while ignoring risk. At the other end of the emotional spectrum,
    when prices are generally falling, fear of loss causes investors to focus
    solely on the possibility of continued price declines to the exclusion of
    investment fundamentals. Regardless of the market environment, many
    investors seek a formula for success. The unfortunate reality is that
    investment success cannot be captured in a mathematical equation or a
    computer program. \\
  }
  {
    有时投资者最大的敌人是他们自己。股票价格上涨时,贪婪的心态使投资者开始投机,
    他们过于乐观,开始高风险赌博,而完全忽视了风险。同样,股票价格下跌时,对亏损
    的恐惧使投资者过分关注股票价格的下跌而非投资本身。不论市场环境如何,都有很多
    投资者寻找所谓的“成功公式”。但不幸的是,投资成功并不是仅靠一个数学公式或一
    段程序就能实现的。
  }
\end{verseparallel}

\begin{verseparallel}
  {
    The first section of this book, chapters 1 through 4, examines some of the
    places where investors stumble. Chapter 1 explores the differences between
    investing and speculation and between successful and unsuccessful investors,
    examining in particular the role of market price in investor behavior.
    Chapter 2 looks at the way Wall Street, with its short-term orientation,
    conflicts of interest, and upward bias, maximizes its own best interests,
    which are not necessarily also those of investors. Chapter 3 examines the
    behavior of institutional investors, who have come to dominate today's
    financial markets. Chapter 4 uses the case study of junk bonds to illustrate
    many of the pitfalls highlighted in the first three chapters. \\
  }
  {
    本书的第一部分{} (1 到 4 章)分析了大多数投资者会在哪摔倒。第一章分析了投资和
    投机的区别,以及成功投资者和失败投资者的区别,以及市场价格如何影响投资者的行
    为。第二章分析了华尔街的特点,他们短视,投资者的利益并不一致,因此只会最大化
    自己的利益。第三章分析如今主导市场的机构投资者的行为。第四章利用垃圾债券的例
    子说明前三章中提到的投资者的种种问题。
  }
\end{verseparallel}

\begin{verseparallel}
  {
    The rapid growth of the market for newly issued junk bonds was only made
    possible by the complicity of investors who suspended disbelief. Junk-bond
    buyers greedily accepted promises of a free lunch and willingly adopted new
    and unproven methods of analysis. Neither Wall Street nor the institutional
    investment community objected vocally to the widespread proliferation of
    these flawed instruments. \\
  }
  {
    新兴垃圾债券市场的快速发展完全是那些停止怀疑的投资者的贡献。垃圾债券的买家贪
    婪的相信免费午餐,愿意使用新的未经证实的分析方法进行投资。无论是华尔街还是机
    构投资者,都没有发声反对这些有问题的垃圾债券的快速扩张。
  }
\end{verseparallel}

\begin{verseparallel}
  {
    Investors must recognize that the junk-bond mania was not a
    once-in-a-millennium madness but instead part of the historical ebb and flow
    of investor sentiment between greed and fear. The important point is not
    merely that junk bonds were flawed (although they certainly were) but that
    investors must learn from this very avoidable debacle to escape the next
    enticing market fad that will inevitably come along. \\
  }
  {
    投资者必须认识到,对垃圾债券的狂热并不是罕见的,其实这只是投资者在贪婪和恐惧
    之间摇摆的一部分。重点并不是垃圾债券有问题{} (虽然确实如此),而是投资者必须从
    中吸取教训,在下次遇到类似情况时能及时远离。
  }
\end{verseparallel}

\begin{verseparallel}
  {
    A second important reason to examine the behavior of other investors and
    speculators is that their actions often inadvertently result in the creation
    of opportunities for value investors. Institutional investors, for example,
    frequently act as lumbering behemoths, trampling some securities to large
    discounts from underlying value even as they ignore or constrain themselves
    from buying others. Those they decide to purchase they buy with gusto; many
    of these favorites become significantly overvalued, creating selling (and
    perhaps short-selling) opportunities. \\
  }
  {
    分析其他投资者和投机者行为的第二个原因,是他们的做法会在不经意间给价值投资者
    创造机会。机构投资者经常像笨拙的庞然大物,他们把一些证券的价格打压到比内在价
    值还要低很多,而他们兴致勃勃买入的那些股票,很多都已经明显高估,给价值投资者
    创造卖出甚至是卖空的机会。
  }
\end{verseparallel}

\begin{verseparallel}
  {
    Herds of individual investors acting in tandem can similarly bid up the
    prices of some securities to crazy levels, even as others are ignored or
    unceremoniously dumped. Abetted by Wall Street brokers and investment
    bankers, many individual as well as institutional investors either ignore or
    deliberately disregard underlying business value, instead regarding stocks
    solely as pieces of paper to be traded back and forth. \\
  }
  {
    个人投资者群体同样可以将某些股票的价格推到难以置信的高位,就像其他人完全不存
    在一样。在华尔街券商和投行人员的怂恿下,个人投资者和机构投资者都忽视甚至完全
    不考虑公司的内在价值,只把股票看成是用来交易的小纸片。
  }
\end{verseparallel}

\begin{verseparallel}
  {
    The disregard for investment fundamentals sometimes affects the entire stock
    market. Consider, for example, the enormous surge in share prices between
    January and August of 1987 and the ensuing market crash in October of that
    year. In the words of William Ruane and Richard Cunniff, chairman and
    president of the Sequoia Fund, Inc., `Disregarding for the moment whether
    the prevailing level of stock prices on January 1, 1987 was logical, we are
    certain that the value of American industry in the aggregate had not
    increased by 44\% as of August 25. Similarly, it is highly unlikely that the
    value of American industry declined by 23\% on a single day, October 19.' \\
  }
  {
    对基本面的漠视有时会影响整个股票市场。例如 1987 年 1--7 月间美股的大幅上涨以
    及之后 10 月的崩盘。就像红杉基金董事长比尔·鲁安和理查德·康尼夫所说:“不管
    1987 年 1 月的股价水平是否合理,但我们确信到 8 月美国的工业没有增长 44\%,当
    然在 10 月 19 日那一天也不可能减少了 23\%。”
  }
\end{verseparallel}

\begin{verseparallel}
  {
    Ultimately investors must choose sides. One side—the wrong choice—is a
    seemingly effortless path that offers the comfort of consensus. This course
    involves succumbing to the forces that guide most market participants,
    emotional responses dictated by greed and fear and a short-term orientation
    emanating from the relative-performance derby. Investors following this road
    increasingly think of stocks like sowbellies, as commodities to be bought
    and sold. This ultimately requires investors to spend their time guessing
    what other market participants may do and then trying to do it first. The
    problem is that the exciting possibility of high near-term returns from
    playing the stocks-as-pieces-of-paper-that-you-trade game blinds investors
    to its foolishness. \\
  }
  {
    最终,投资者必须二选一。错误的选择是一条看上去毫不费力的路径,也能获得其他投
    资者的认同。因为这一方包括大多数被市场引导的投资者,他们的情绪不断在贪婪和恐
    惧之间摇摆,同时受到短期的相对投资业绩的影响。站在这一方的投资者觉得股票和腌
    猪肉没什么区别,就像是一种用于买卖的商品。站在这边的结果是,需要投资者事先猜
    测市场上的其他人可能做什么,然后试着捷足先登领先其他人一步。但问题是,在短期
    高回报的诱惑下去玩这种买卖股票纸片的游戏,很可能让投资者变得盲目和愚蠢。
  }
\end{verseparallel}

\begin{verseparallel}
  {
    The correct choice for investors is obvious but requires a level of
    commitment most are unwilling to make. This choice is known as fundamental
    analysis, whereby stocks are regarded as fractional ownership of the
    underlying businesses that they represent. One form of fundamental
    analysis—and the strategy that I recommend—is an investment approach known
    as value investing. \\
  }
  {
    投资者的正道很明显,但需要付出一些努力,而大多数人却不愿意。这个正道就是基本
    面分析,把股票当成公司的一部分。我推荐的投资策略,也属于基本面分析的一种,就
    是价值投资。
  }
\end{verseparallel}

\begin{verseparallel}
  {
    There is nothing esoteric about value investing. It is simply the process of
    determining the value underlying a security and then buying it at a
    considerable discount from that value. It is really that simple. The
    greatest challenge is maintaining the requisite patience and discipline to
    buy only when prices are attractive and to sell when they are not, avoiding
    the short-term performance frenzy that engulfs most market participants. \\
  }
  {
    价值投资并不高大上。判断一只股票的内在价值,只有在其基础上再打一个合理的折扣
    价格才买入,就是这么简单。价值投资最大的挑战是保持耐心,遵守纪律,只有股价有
    吸引力时才买入,没有吸引力时就卖出,这样就能避免对短期投资表现的狂热追捧,这
    种追捧吞没了大多数市场参与者。
  }
\end{verseparallel}

\begin{verseparallel}
  {
    The focus of most investors differs from that of value investors. Most
    investors are primarily oriented toward return, how much they can make, and
    pay little attention to risk, how much they can lose. \\
  }
  {
    价值投资者关心的和大多数投资者不一样。大多数投资者主要关注收益,也就是他们能
    赚多少,而几乎不考虑风险,不去管他们会赔多少。
  }
\end{verseparallel}
\begin{verseparallel}
  {
    Institutional investors, in particular, are usually evaluated—and therefore
    measure themselves on the basis of relative performance compared to the
    market as a whole, to a relevant market sector, or to their peers. \\
  }
  {
    机构投资者经常会比较自己和大盘、特定板块或者其他机构投资者的相对业绩,并据此
    评价自己的表现。
  }
\end{verseparallel}
\begin{verseparallel}
  {
    Value investors, by contrast, have as a primary goal the preservation of
    their capital. It follows that value investors seek a margin of safety,
    allowing room for imprecision, bad luck, or analytical error in order to
    avoid sizable losses over time. A margin of safety is necessary because
    valuation is an imprecise art, the future is unpredictable, and investors
    are human and do make mistakes. It is adherence to the concept of a margin
    of safety that best distinguishes value investors from all others, who are
    not as concerned about loss. \\
  }
  {
    与之相反,价值投资者的首要目标是保住自己的本金。价值投资者会寻找安全边际,这
    样才能保证在计算不够精确、霉运、分析错误等情况出现时,自己不会遭受巨大损失。
    安全边际十分必要,因为估值是一门模糊的艺术,未来难以预测,投资者也是人,也会
    犯错。安全边际很好的将价值投资者和其他不那么重视本金安全的人区分开来。
  }
\end{verseparallel}
\begin{verseparallel}
  {
    If investors could predict the future direction of the market, they would
    certainly not choose to be value investors all the time. Indeed, when
    securities prices are steadily increasing, a value approach is usually a
    handicap; out-of-favor securities tend to rise less than the public's
    favorites. When the market becomes fully valued on its way to being
    overvalued, value investors again fare poorly because they sell too soon. \\
  }
  {
    如果投资者能够预测未来市场的走向,他们当然不想一直做价值投资者。实际上,当股
    票市场稳步上涨时,价值投资方法反倒会影响你的收益,因为价值投资者选择的那些不
    受追捧的股票涨得比市场上受追捧的股票要慢。当市场价格已经反映了公司的内在价值,
    向高估发展时,价值投资者同样表现不佳,因为他们卖的太早了。
  }
\end{verseparallel}
\begin{verseparallel}
  {
    The most beneficial time to be a value investor is when the market is
    falling. This is when downside risk matters and when investors who worried
    only about what could go right suffer the consequences of undue optimism.
    Value investors invest with a margin of safety that protects them from large
    losses in declining markets. \\
  }
  {
    成为价值投资者最有利的时期是市场下行期。这个时期风险很重要,同时投资者会担心
    过度乐观导致的后果。价值投资者投资时会考虑安全边际,这可以保护他们在下行市场
    中免受巨大损失。
  }
\end{verseparallel}
\begin{verseparallel}
  {
    Those who can predict the future should participate fully, indeed on margin
    using borrowed money, when the market is about to rise and get out of the
    market before it declines. Unfortunately, many more investors claim the
    ability to foresee the market's direction than actually possess that
    ability. (I myself have not met a single one.) Those of us who know that we
    cannot accurately forecast security prices are well advised to consider
    value investing, a safe and successful strategy in all investment
    environments. \\
  }
  {
    那些能够预测市场的投资者在股市上涨时应该加杠杆满仓,在股市下跌前清仓。不幸的
    是,大多数投资者所声称的预测能力比他们实际拥有的预测能力大得多{} (我个人还没
    遇到任何一个相反的个例)。对于我们这种知道自己不能精确预测股票价格的投资者,最
    好采用价值投资策略,这种策略在各种投资环境中都是安全的,同时也能获得不错的收益。
  }
\end{verseparallel}
\begin{verseparallel}
  {
    The second section of this book, chapters 5 through 8, explores the
    philosophy and substance of value investing. Chapter 5 examines why most
    investors are risk averse and discusses the investment implications of this
    attitude. \\
  }
  {
    本书的第二部分{} (5 到 8 章),探讨了价值投资的原理和本质。第五章讨论了为什么
    大多数投资者厌恶风险,以及这种态度对投资的影响。
  }
\end{verseparallel}
\begin{verseparallel}
  {
    Chapter 6 describes the philosophy of value investing and the meaning and
    importance of a margin of safety. Chapter 7 considers three important
    underpinnings to value investing: a bottom-up approach to investment
    selection, an absolute-performance orientation, and analytical emphasis on
    risk as well as return. Chapter 8 demonstrates the principal methods of
    securities valuation used by value investors. \\
  }
  {
    第六章介绍了价值投资的基本原理和安全边际的重要性。第七章介绍了三个价值投资的
    重要基础:自顶而上进行投资选择的方法,绝对收益导向,对风险和回报的定量分析。
    第八章介绍了价值投资者使用的估值方法。
  }
\end{verseparallel}
\begin{verseparallel}
  {
    The third section of this book, chapters 9 through 14, describes the
    value-investment process, the implementation of a value-investment
    philosophy. Chapter 9 explores the research and analytical process, where
    value investors get their ideas and how they evaluate them. Chapter 10
    illustrates a number of different value-investment opportunities ranging
    from corporate liquidations to spinoffs and risk arbitrage. Chapters 11 and
    12 examine two specialized value-investment niches: thrift conversions and
    financially distressed and bankrupt securities, respectively. Chapter 13
    highlights the importance of good portfolio management and trading
    strategies. Finally, Chapter 14 provides some insight into the possible
    selection of an investment professional to manage your money. \\
  }
  {
    本书的第三部分{} (9 到 14 章),介绍了从价值投资哲学出发,进行价值投资的具体
    过程。第九章探讨了价值投资的研究和分析过程,包括价值投资者的灵感从哪来,以及
    如何对这些公司估值。第十章介绍了一系列价值投资机会,包括公司清算分拆和风险套
    利。第十一章和第十二章介绍了两个具体的价值投资领域,节俭转换和破产证券。第十
    三章强调了良好资产组合管理和交易策略的重要性。第十四章提供了一些专业投资者替
    你管理资产的建议。
  }
\end{verseparallel}
\begin{verseparallel}
  {
    The value discipline seems simple enough but is apparently a difficult one
    for most investors to grasp or adhere to. As Buffett has often observed,
    value investing is not a concept that can be learned and applied gradually
    over time. It is either absorbed and adopted at once, or it is never truly
    learned. \\
  }
  {
    价值投资的纪律看起来很简单,但对大多数投资者而言,能够坚持这些纪律还是很难的。
    就像沃伦·巴菲特提到的,价值投资不是那种你需要渐悟才能理解的。价值投资要么秒懂,
    要么永远也学不会。
  }
\end{verseparallel}
\begin{verseparallel}
  {
    I was fortunate to learn value investing at the inception of my investment
    career from two of its most successful practitioners: Michael Price and the
    late Max L. Heine of Mutual Shares Corporation. While I had been fascinated
    by the stock market since childhood and frequently dabbled in the market as
    a teenager (with modest success), working with Max and Mike was like being
    let in on an incredibly valuable secret. How naive all of my previous
    investing suddenly seemed compared with the simple but incontrovertible
    logic of value investing. Indeed, once you adopt a value-investment
    strategy, any other investment behavior starts to seem like gambling. \\
  }
  {
    我很幸运,在自己投资事业起步的时候,就能从两位前辈身上学到价值投资,他们是米
    切尔·普莱斯和互惠基金的麦克斯·海宁。虽然我从小就沉迷股票市场,而且作为一个年
    轻人在股市中取得了一点成绩,但和这两位前辈一起工作就像身处一个无尽的宝藏。和
    简单但无可争议的价值投资相比,我之前的投资都显得无比幼稚。实际上,一旦你接受
    价值投资,其他投资行为看起来都像是赌博。
  }
\end{verseparallel}
\begin{verseparallel}
  {
    Throughout this book I criticize certain aspects of the investment business
    as currently practiced. Many of these criticisms of the industry appear as
    generalizations and refer more to the pressures brought about by the
    structure of the investment business than the failings of the individuals
    within it. \\
  }
  {
    我对目前投资行业某些方面的批评贯穿本书。这些批评更多的是针对整个行业的情况,
    而非针对具体的人和个例,实际上这些问题更多的是由目前投资行业的结构导致的。
  }
\end{verseparallel}
\begin{verseparallel}
  {
    I also give numerous examples of specific investments throughout this book.
    Many of them were made over the past nine years by my firm for the benefit
    of our clients and indeed proved quite profitable. The fact that we made
    money on them is not the point, however. My goal in including them is to
    demonstrate the variety of value-investment opportunities that have arisen
    and become known to me during the past decade; an equally long and rich list
    of examples failed to make it into the final manuscript. \\
  }
  {
    在本书中我会举很多例子,其中很多都是过去九年内我为客户着想所做的投资决定,也
    确实给我的客户带来了丰厚回报。我并不单单想说我们在这些投资中赚了钱。我把它们
    写入本书的原因,是想说明过去十年中我其实发现了各种各样的价值投资机会,当然也
    有相当多的例子没有写入本书的终稿。
  }
\end{verseparallel}
\begin{verseparallel}
  {
    I find value investing to be a stimulating, intellectually challenging, ever
    changing, and financially rewarding discipline. I hope you invest the time
    to understand why I find it so in the pages that follow. \\
  }
  {
    我觉得价值投资能激发灵感,同时很有挑战性,价值投资也在持续进化,是能带来良好
    回报的纪律性投资方法。我希望你在本书其余部分的阅读中能理解我的这些想法。
  }
\end{verseparallel}


%%% Local Variables:
%%% TeX-master: "../master"
%%% End:

\documentclass[12pt, a4paper, UTF8, fontset=adobe, oneside]{ctexbook} % oneside 去掉所有空白页

\setmainfont{Times New Roman} % 英文用 Times New Roman 字体
\linespread{1.3} % 行距设置
\setcounter{secnumdepth}{3} % 层次为 3 以上的标题生成序号

%% 宏包
\usepackage{amsmath} % AMS 数学宏包
\usepackage{amssymb} % AMS 字体宏包
\usepackage{fancyhdr} % 设置页眉页脚宏包
\usepackage{geometry} % 设置页边距宏包
\usepackage{xcolor} % 颜色宏包
\usepackage[listings,breakable]{tcolorbox} % 彩色盒子宏包 代码宏包
\usepackage{enumitem} % 枚举设置宏包
\usepackage{tikz} % 画图宏包
\usetikzlibrary{positioning} % tikz 相对位置 library
\usetikzlibrary{calc} % tikz 计算 library
\usepackage{booktabs} % 表格宏包
\usepackage{dsfont} % 数字粗体宏包
\usepackage{titlesec} % 字体大小设置宏包
\usepackage{parallel} % 双栏排版宏包
\usepackage{hyperref} % 交叉引用宏包 colorlinks启用彩色模式 参考文献引用为紫红色
\usepackage{changepage} % 调整 tikz 图左右边距宏包
\usepackage[hang,flushmargin]{footmisc}

% 宏包设置
% 页眉页脚样式
\pagestyle{fancy} % 页面样式采用fancyhdr宏包中的fancy
\fancyhf{} % 去掉页眉
\cfoot{\thepage} % 页脚中间显示页码
\renewcommand{\headrulewidth}{0pt} % 去掉页眉的横线
% 页边距设置
\geometry{top = 2.54cm, bottom = 2.54cm, left = 3.18cm, right = 3.18cm}
% 章节样式设置
\CTEXsetup[name={第,章},number={\arabic{chapter}}]{chapter}
% 文档设置
\renewcommand\contentsname{目录} % 中文 目录
\renewcommand\bibname{参考文献} % 中文 参考文献
% 清华紫
\definecolor{THU}{RGB}{111, 23, 135}
\definecolor{水绿}{RGB}{140, 194, 105} % 理想买点颜色
\definecolor{霁青}{RGB}{99, 187, 208} % 合理估值下限颜色
\definecolor{玫瑰紫}{RGB}{186, 47, 123} % 三年后合理估值颜色
% 交叉引用宏包设置
\hypersetup{colorlinks=true,linkcolor=THU,citecolor=THU}
% itemize 行距设置
\usepackage{enumitem}
\setenumerate[1]{itemsep=0pt,partopsep=0pt,parsep=\parskip,topsep=0pt}
\setitemize[1]{itemsep=0pt,partopsep=0pt,parsep=\parskip,topsep=0pt}
% 字体大小设置
\titleformat*{\section}{\LARGE\bfseries}
\titleformat*{\subsection}{\Large\bfseries}
\titleformat*{\subsubsection}{\large\bfseries}
% 设置 paragraph 左/前/后距离
\titlespacing{\paragraph}{0pt}{0.3\baselineskip}{1em}

% tcolorbox 样式设置
\newtcolorbox{redbox}[2][]{colback=yellow!10,colframe=red!75!black,coltitle=white,fonttitle=\bfseries,fontupper=\kaishu,title=#2,#1,breakable} % 红色
\newtcolorbox{RCbox}[2][]{colback=yellow!10,colframe=red!75!black,coltitle=white,fonttitle=\bfseries,fontupper=\kaishu,title=#2,#1,center
  title, center upper,breakable} % 红色居中
\newtcolorbox{magbox}[2][]{colback=yellow!10,colframe=magenta!75!black,coltitle=white,fonttitle=\bfseries,fontupper=\kaishu,title=#2,#1} % 紫红色
\newtcolorbox{THUbox}[2][]{colback=yellow!10,colframe=THU!75!black,coltitle=white,fonttitle=\bfseries,fontupper=\kaishu,title=#2,#1,breakable} % 紫罗兰色
\newtcolorbox{THUCbox}[2][]{colback=yellow!10,colframe=THU!75!black,coltitle=white,fonttitle=\bfseries,fontupper=\kaishu,title=#2,#1,center title,center upper,breakable} % 紫罗兰色 居中
\newtcolorbox{purbox}[2][]{colback=yellow!10,colframe=purple!75!black,coltitle=white,fonttitle=\bfseries,fontupper=\kaishu,title=#2,#1,center title,center upper} % 紫色

% itemize/enumerate 样式设置
\setlist[enumerate]{label={\arabic*.},leftmargin=2.5em,align=left,topsep=0em,itemsep=-0.5em,labelsep=-1em,before=\vspace{2pt},after=\vspace{2pt}}
\setlist[itemize]{leftmargin=2.5em,align=left,topsep=0em,itemsep=-0.5em,labelsep=-1em,before=\vspace{2pt},after=\vspace{2pt}}

% 引用参考文献时提高位置
\newcommand{\citerb}[1]{\raisebox{1pt}{\cite{#1}}}

% 设置页码格式
\fancypagestyle{plain}{%
  \fancyhf{} % clear all header and footer fields
  \fancyfoot[C]{\fontsize{9pt}{9pt}\selectfont\thepage} % except the center
  \renewcommand{\headrulewidth}{0pt}
  \renewcommand{\footrulewidth}{0pt}}
\pagestyle{plain}

% 并行排版 + 脚注形式
\newenvironment{verseparallel}[2]
{\begin{Parallel}{}{}\footnotesize\parindent=0pt
    \ParallelLText{#1}\ParallelRText{#2}}
  {\end{Parallel}}

\begin{document}
\frontmatter
\begin{titlepage}
\begin{center}

\vspace*{5cm}
% Title
{\huge \bfseries 唐朝老师持仓买卖点}\\[0.4cm]

\vspace{12cm}

{\large 江浩} \\[1cm]
{\large \today}

\end{center}
\end{titlepage}

{
\hypersetup{linkcolor=black} % 目录链接为黑色
\pagenumbering{Roman} % 页码编号为大写罗马数字
\tableofcontents % 目录
}

\mainmatter% 正文部分 重新编号
\pagenumbering{arabic} % 页码编号为阿拉伯数字

\chapter{洋河股份}
\noindent
\begin{figure}[htbp]
\begin{adjustwidth*}{-7em}{}
\begin{tikzpicture}
  % 各点坐标
  \coordinate (lixiang) at (0,5.02);
  \coordinate (dangqian) at (0,5.58);
  \coordinate (helixia) at (0,5.97);
  \coordinate (sannian) at (0,10.03);

  % 坐标轴
  \draw[line width=2pt,->] (0,0) -- (15,0);
  \draw[line width=2pt,->] (0,0) -- (0,18);

  \draw[水绿,line width=1pt,dashed] (lixiang) -- ($ (lixiang) + (15,0) $);
  \node [left=0.1em of lixiang](lixiang_text){\color{水绿}{理想买点 139.35}};

  \draw[line width=2pt,solid] (dangqian) -- ($ (dangqian) + (15,0) $);
  \node [left=0.1em of dangqian](dangqian_text){当前股价 155.00};

  \draw[霁青,line width=1pt,dashed] (helixia) -- ($ (helixia) + (15,0) $);
  \node [left=0.1em of helixia](helixia_text){\color{霁青}{合理估值下限 165.89}};

  \draw[玫瑰紫,line width=1pt,dashed] (sannian) -- ($ (sannian) + (15,0) $);
  \node [left=0.1em of sannian](sannian_text){\color{玫瑰紫}{三年后合理估值 278.7}};
\end{tikzpicture}
\end{adjustwidth*}
\end{figure}

\newpage
洋河股份目前总股本为 15.06988 亿,2022 年预计净利润 100 亿,三年后预计净利润 140
亿。三年后合理估值上限的 150\% 为 $140 \times 30 \times 150\% = 6300$ 亿,2022 年
净利润的 50 倍为 5000 亿,一年内卖点取二者较低值即 5000 亿。

\begin{table}[htbp]
  \caption{洋河股份市值和股价对应关系}
  \begin{center}
  \begin{tabular}{ccc}
    \toprule
    说明 & 市值(亿元)& 股价(元)  \\
    \midrule
    理想买点 & 2100 & 139.35  \\
    当前合理估值下限 & 2500 & 165.89  \\
    当前合理估值中值 & 2750 & 182.48  \\
    当前合理估值上限 & 3000 & 199.07  \\
    三年后合理估值 & 4200 & 278.70  \\
    一年内第一卖点 & 5000 & 331.78  \\
    一年内第二卖点 & 5500 & 364.97  \\
    一年内第三卖点 & 6000 & 398.15  \\
    \bottomrule
  \end{tabular}
\end{center}
\end{table}

\bibliographystyle{thubib}
\bibliography{refs}
\end{document}

%%% Local Variables:
%%% TeX-master: t
%%% End:
